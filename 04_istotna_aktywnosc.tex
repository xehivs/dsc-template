\section{Informacja o wykazywaniu się istotną aktywnością naukową albo artystyczną realizowaną w więcej niż jednej uczelni, instytucji naukowej lub instytucji kultury, w szczególności zagranicznej}

%Pracę naukową rozpocząłem na ostatnim roku studiów magisterskich, realizując projekt przewidywania występowania sztormów na Morzu Bałtyckim, w oparciu o samodzielnie skonstruowany strumień serii czasowej dla zadanej, interpolowanej siatki punktów próbkowania stanu pogody zbudowanej na podstawie odczytów stacji meteorologicznych. W 2013 roku rozpocząłem studia doktoranckie na Wydziale Elektroniki, Politechniki Wrocławskiej, skupiając się na tematyce przetwarzania wielowymiarowych sygnałów cyfrowych w kontekście ich segmentacji, klasteryzacji i klasyfikacji. W 2015 roku uzyskałem zatrudnienie na stanowisku asystenta naukowo-dydaktycznego w Katedrze Systemów i Sieci Komputerowych, gdzie też w 2017 roku złożyłem pracę doktorską w tematyce reprezentacji i analizy danych wielowymiarowych. Stopień doktora nauk technicznych w dyscyplinie informatyka (z wyróżnieniem) uzyskałem zgodnie z uchwałą Rady Wydziału Elektroniki Politechniki Wrocławskiej, 21 czerwca 2017 roku. 

Po zatrudnieniu na stanowisku adiunkta, w październiku 2017 roku, zmieniłem główne zainteresowania badawcze, skupiając się na klasyfikacji danych niezbalansowanych i przetwarzaniu strumieni danych, ze szczególnym uwzględnieniem wsadowego paradygmatu przetwarzania}. 

Dalsze prace starałem się realizować w sposób, który pozwala na retencję wiedzy i stosowanych metod, przykładając szczególną wagę do replikowalności eksperymentów i otwartości oprogramowania opracowywanego w ramach bieżących prac \emph{Katedry Systemów i Sieci Komputerowych}\sidenote[][-5em]{Większość badań pracowników Katedry z ostatnich pięciu lat publikowana jest również jako oprogramowanie eksperymentalne dostępne na prowadzonym przeze mnie profilu organizacji na GitHub:\\\noindent\url{https://github.com/w4k2}}. Silną współpracę w tym okresie prowadziłem zarówno z~doktorantami zatrudnionymi w~jednostce, z których dwóch uzyskało w~tym roku stopień doktora, jak i z~pracownikami innych zespołów badawczych \emph{Politechniki Wrocławskiej}. 

Szczególnie istotny jest dla mnie aspekt multidyscyplinarności badań, która pozwala na odnajdywanie szerszych zastosowań  proponowanych przeze mnie metod. Wyraża się ona, między innymi, w ścisłej współpracy z \emph{Zespołem Sieci Komputerowych}, gdzie wspólnie z badaczami z~\emph{Instytutu Łączności\sidenote[][-7em]{\\\noindent\emph{Instytut Łączności -- Państwowy Instytut Badawczy}\\\noindent\url{https://itl.waw.pl/}}} budujemy algorytmy wspomagające optymalizację kognitywnych sieci optycznych z wykorzystaniem modeli regresji~\citeM[-4.25em]{Ksi20s}. Wykorzystując zdobytą wcześniej wiedzę z zakresu przetwarzania cyfrowych sygnałów wielowymiarowych, angażuję się także w~badania z~zakresu bioinformatyki, gdzie wspólnie z~pracownikami \emph{Katedry Inżynierii Biomedycznej} oraz \emph{School of Optometry and Vision Science}\sidenote[][-1em]{\\\noindent\emph{School of Optometry and Vision Science},\\\noindent Brisbane, Australia} i~\emph{Uniwersytetu Medycznego we Wrocławiu} analizujemy potencjał metod uczenia maszyn w zagadnieniu wczesnej detekcji jaskry~\citeM{Sul21b}.

Po zakończeniu doktoratu nawiązałem też silną współpracę z pracownikami \emph{Zakładu Systemów Teleinformatycznych Wydziału Telekomunikacji, Informatyki i Elektrotechniki, Politechniki Bydgoskiej}. W ramach współpracy zrealizowaliśmy międzynarodowy projekt \emph{SocialTruth}, finansowany ze środków programu \emph{EU Horizon 2020}, prowadzony w konsorcjum z 11 państw \emph{Unii Europejskiej}\sidenote{\emph{SocialTruth}\\\url{http://socialtruth.eu}}. W latach 2019--2021 byłem zatrudniony w tym projekcie, publikując cztery prace naukowe, a dwie kolejne oczekują obecnie na recenzje.

Osiągnięcia poczynione w toku realizacji projektu \emph{SocialTruth} pozwoliły nam również opracować wniosek projektowy w ramach programu \textsc{infostrateg i}\sidenote{\url{https://www.gov.pl/web/ncbr/infostrateg-i-konkurs}}, poświęcony zagadnieniu klasyfikacji fake news w~języku polskim, który uzyskał maksymalną punktację \textsc{ncb}i\textsc{r}, zdobył finansowanie na poziomie ponad ośmiu milionów złotych~\sidenote{\emph{System Wykrywania Dezinformacji Metodami Sztucznej Inteligencji
}\\\noindent\url{https://www.kssk.pwr.edu.pl/projects/swarog}} i w którym od grudnia 2021 roku pełnię rolę kierownika badawczo-rozwojowego. Projekt \emph{SWAROG} realizowany jest w konsorcjum z \emph{Politechniką Bydgoską} i przedsiębiorstwem \emph{Matic SA}.

Wspólna praca w międzynarodowym konsorcjum -- m. in. podczas spotkań projektowych w Paryżu i Rzymie -- pozwoliła mi także na nawiązanie współpracy z innymi europejskimi ośrodkami w zakresie wspólnych publikacji. Wspólne badania z zakresu klasyfikacji \emph{fake news} z zespołami \emph{National Technical University of Athens\sidenote[][-3em]{\\\noindent\emph{National Technical University of Athens},\\\noindent Ateny, Grecja}} i \emph{Universidad de Burgos\sidenote{\\\noindent\emph{Universidad de Burgos},\\\noindent Burgos, Hiszpania}} doprowadziły do opracowania pracy przeglądowej z zakresu metod detekcji źródeł dezinformacji~\citeM[.5em]{Cho21}. W ubiegłym roku pełniłem też rolę członka komisji doktorskiej przewodu Nuno Basutro z \emph{Universidad de Burgos}, który podczas swoich studiów doktoranckich odbył staż na \emph{Politechnice Wrocławskiej}.

Wymiana doświadczeń z zakresu projektowania eksperymentów rozpoznawania wzorców -- ze szczególnym uwzględnieniem metod statystycznego testowania hipotez -- podjęta wspólnie z zespołami \emph{Politechniki Śląskiej} i \emph{University of Granada}\sidenote{\\\noindent\emph{University of Granada},\\\noindent Granada, Hiszpania} pozwoliły z kolei na opracowanie pracy przeglądowej dotyczącej dobrych praktyk eksperymentowania z~algorytmami klasyfikacji~\citeM{Sta21}. Wprowadzamy tym zarówno zbiór wytycznych poprawnej ewaluacji, jak i prezentujemy praktyczne przykłady powszechnych, acz błędnych strategii oceny, które często prowadzą do uzyskiwania niejednoznacznych wniosków z badań.\vspace{1em}

\noindent W toku swojej pracy naukowej odbyłem także trzy krótko i średnioterminowe staże naukowe:

\begin{itemize}
	\item Dwa staże naukowe zrealizowałem w ramach współpracy z \emph{Universidad del Pais Vasco}\sidenote[][-2em]{\\\noindent\emph{Universidad del Pais Vasco}\\\noindent San Sebastian, Hiszpania}, które zakończyły się nawiązaniem trwałej współpracy badawczej z zespołem \emph{Group Faculty of Informatics} kierowanym przez profesora Manuela Granę~\citeM[-1.5em]{Ksi17p}. Współpraca ta wyraża się zarówno we wspólnych pracach badawczych, jak i w wymianie doświadczeń projektowych, której podsumowaniem może być wykład, który wygłosiłem na zaproszenie podczas konferencji \emph{CybSPEED}'22. Współpraca nawiązana na wczesnym etapie pracy badawczej z \emph{Universidad del Pais Vasco} była też bardzo ważnym czynnikiem w toku realizacji mojego doktoratu, ponieważ to podczas stażu Borja Ayerdi z zespołu prof. Manuela Grany realizowałem swoje pierwsze badania z zakresu przetwarzania obrazów nadwidmowych. 
	
	\item Trzeci staż naukowy zrealizowałem w \emph{Virginia Commonwealth University}\sidenote[][-9.5em]{\\\noindent\emph{Virginia Commonwealth University}\\\noindent Richmond, VA, USA}, co pozwoliło mi na rozwinięcie wstępnych koncepcji badań z zakresu klasyfikacji danych trudnych~\citeC[-0em]{C10}. W ramach rozwijania zakresu współpracy analizujemy aktualnie tematykę \emph{wyjaśnialnej sztucznej inteligencji} (ang. \emph{explainable AI}) dla zagadnień przetwarzania strumieni danych oraz homogeniczne metody dywersyfikacji puli klasyfikatorów oparte o rozkłady niejednostajne.

\end{itemize}
 