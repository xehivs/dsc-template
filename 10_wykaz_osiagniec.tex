\begin{fullwidth}

\chapter{Wykaz osiągnięć naukowych albo artystycznych, stanowiących znaczny wkład w rozwój określonej dyscypliny}

	
\section{INFORMACJA O OSIĄGNIĘCIACH NAUKOWYCH ALBO ARTYSTYCZNYCH O KTÓRYCH MOWA W ART. 219 UST. 1. PKT 2 USTAWY}

\subsection{Monografia naukowa}
---

\subsection{\textbf{Cykl powiązanych tematycznie artykułów naukowych, zgodnie z art. 219 ust. 1. pkt 2b Ustawy; pt:}}

\vspace{1em}
\begin{center}
	\LARGE
	\textbf{Projektowanie algorytmów rozpoznawania wzorców dla~zadania klasyfikacji trudnych danych}
\end{center}\vspace{1em}

\noindent Wszystkie publikacje pochodzą z okresu po uzyskaniu stopnia doktora. Informacje dot. liczby punktów MEiN, współczynnika IF oraz liczby cytowań  oddają \textbf{stan na dzień 27. sierpnia 2022 r.} zgodnie z bazami publikacji naukowych:

\begin{itemize}
	\item[\textsc{WoS}] \emph{Web of Science}
	\url{https://www.webofscience.com/wos/author/record/1886494}	
	\item[\textsc{Sco}] \emph{Scopus}
	\url{https://www.scopus.com/authid/detail.uri?authorId=56206176100}
	\item[\textsc{GSc}] \emph{Google Scholar}	
	\url{https://scholar.google.com/citations?user=YSM30D8AAAAJ}
\end{itemize}

\newpage

\end{fullwidth}

%
% C1
%
\marginnote[1em]{
% MARGIN TAB
\color{red}
\begin{tabular}{rccc}
\toprule
& \textsc{WoS} & \textsc{Sco} & \textsc{GSc}\\
l. cytowań & --- & --- & --- \\\\
\multicolumn{3}{r}{\color{black}\emph{Szacowany udział}} & \color{black}70\%\\
\multicolumn{3}{r}{\color{black}\emph{Impact Factor}} & \color{black}8.139\\
\multicolumn{3}{r}{\color{black}\emph{l. punktów \textsc{me}i\textsc{n}}} & \color{black}200\\\\

%\multicolumn{4}{r}{\includegraphics[width=1.5cm]{WoS.png}}
\end{tabular}

% BASE TAB
}\noindent\begin{tabular}{lllll@{}}
\toprule
\color{red}[C1] & \multicolumn{4}{p{28em}}{\fullcite{C1}} \\
 & \multicolumn{4}{p{30em}}{
 {\footnotesize\textsc{CRediT}}: 
 \mybox{Conceptualization} \mybox{Software} \mybox{Validation} \mybox{Investigation} \mybox{Writing - Original Draft} \mybox{Writing - Review \& Editing} \mybox{Visualization} \mybox{Supervision}}\\

%\\\\\bottomrule
\end{tabular}
\vspace{.5em}

%
% C2
%
\marginnote[1em]{
% MARGIN TAB
\color{red}
\begin{tabular}{rccc}
\toprule
& \textsc{WoS} & \textsc{Sco} & \textsc{GSc}\\
l. cytowań & --- & --- & --- \\\\
\multicolumn{3}{r}{\color{black}\emph{Szacowany udział}} & \color{black}100\%\\
\multicolumn{3}{r}{\color{black}\emph{Impact Factor}} & \color{black}5.086\\
\multicolumn{3}{r}{\color{black}\emph{l. punktów \textsc{me}i\textsc{n}}} & \color{black} 70\\\\

%\multicolumn{4}{r}{\includegraphics[width=1.5cm]{WoS.png}}

\end{tabular}
% BASE TAB
}\noindent\begin{tabular}{lllll@{}}
\toprule
\color{red}[C2] & \multicolumn{4}{p{28em}}{\fullcite{C2}}  \\
 & \multicolumn{4}{p{30em}}{
 {\footnotesize\textsc{CRediT}}: 
 \mybox{Conceptualization} \mybox{Methodology} \mybox{Software} \mybox{Validation} \mybox{Formal Analysis} \mybox{Investigation} \mybox{Resources} \mybox{Data Curation} \mybox{Writing - Original Draft} \mybox{Writing - Review \& Editing} \mybox{Visualization}}\\
%\\\\\bottomrule
\end{tabular}
\vspace{2.5em}

%
% C3
%
\marginnote[1em]{
% MARGIN TAB
\color{red}
\begin{tabular}{rccc}
\toprule
& \textsc{WoS} & \textsc{Sco} & \textsc{GSc}\\
l. cytowań & 2 & 1 & 4 \\\\
\multicolumn{3}{r}{\color{black}\emph{Szacowany udział}} & \color{black}70\%\\
\multicolumn{3}{r}{\color{black}\emph{Core}} & \color{black}B\\
\multicolumn{3}{r}{\color{black}\emph{l. punktów \textsc{me}i\textsc{n}}} & \color{black} 140\\\\

%\multicolumn{4}{r}{\includegraphics[width=1.5cm]{WoS.png}}

\end{tabular}
% BASE TAB
}\noindent\begin{tabular}{lllll@{}}
\toprule
\color{red}[C3] & \multicolumn{4}{p{28em}}{\fullcite{C3}}  \\
 & \multicolumn{4}{p{30em}}{
 {\footnotesize\textsc{CRediT}}: 
 \mybox{Conceptualization} \mybox{Validation} \mybox{Formal Analysis} \mybox{Investigation} \mybox{Resources} \mybox{Writing - Original Draft} \mybox{Writing - Review \& Editing} \mybox{Visualization} \mybox{Supervision}}\\

%\\\\\bottomrule
\end{tabular}
\vspace{.5em}

%
% C4
%
\marginnote[1em]{
% MARGIN TAB
\color{red}
\begin{tabular}{rccc}
\toprule
& \textsc{WoS} & \textsc{Sco} & \textsc{GSc}\\
l. cytowań & 2 & 2 & 4 \\\\
\multicolumn{3}{r}{\color{black}\emph{Szacowany udział}} & \color{black}100\%\\
\multicolumn{3}{r}{\color{black}\emph{Impact Factor}} & \color{black}5.719\\
\multicolumn{3}{r}{\color{black}\emph{l. punktów \textsc{me}i\textsc{n}}} & \color{black} 140\\\\

%\multicolumn{4}{r}{\includegraphics[width=1.5cm]{WoS.png}}

\end{tabular}

% BASE TAB
}\noindent\begin{tabular}{lllll@{}}
\toprule
\color{red}[C4] & \multicolumn{4}{p{28em}}{\fullcite{C4}}  \\
 & \multicolumn{4}{p{30em}}{
 {\footnotesize\textsc{CRediT}}: 
 \mybox{Conceptualization} \mybox{Methodology} \mybox{Software} \mybox{Validation} \mybox{Formal Analysis} \mybox{Investigation} \mybox{Resources} \mybox{Data Curation} \mybox{Writing - Original Draft} \mybox{Writing - Review \& Editing} \mybox{Visualization}}\\

%\\\\\bottomrule
\end{tabular}
\vspace{2.5em}
%\newpage

%
% C5
%
\marginnote[1em]{
% MARGIN TAB
\color{red}
\begin{tabular}{rccc}
\toprule
& \textsc{WoS} & \textsc{Sco} & \textsc{GSc}\\
l. cytowań & 6 & 7 & 13 \\\\
\multicolumn{3}{r}{\color{black}\emph{Szacowany udział}} & \color{black}50\%\\
\multicolumn{3}{r}{\color{black}\emph{Core}} & \color{black}A\\
\multicolumn{3}{r}{\color{black}\emph{l. punktów \textsc{me}i\textsc{n}}} & \color{black} 140\\\\

%\multicolumn{4}{r}{\includegraphics[width=1.5cm]{WoS.png}}
\end{tabular}

% BASE TAB
}\noindent\begin{tabular}{lllll@{}}
\toprule
%\color{red}[C5] & \multicolumn{4}{p{28em}}{\fullcite{C5}}  \\
\color{red}[C5] & \multicolumn{4}{p{28em}}{Paweł Ksieniewicz, Paweł Zyblewski, Michał Choraś, Rafał Kozik, Agata Giełczyk, Michał Woźniak, \emph{"Fake News Detection from Data Streams"}. W: \emph{2020 International Joint Conference on Neural Networks (IJCNN)}. 2020, s. 1-8. \textsc{doi}: \verb|\url{10.1109/IJCNN48605.2020.9207498}}  \\
 & \multicolumn{4}{p{30em}}{
 {\footnotesize\textsc{CRediT}}: 
 \mybox{Conceptualization} \mybox{Methodology} \mybox{Software} \mybox{Validation} \mybox{Investigation} \mybox{Writing - Original Draft} \mybox{Writing - Review \& Editing} \mybox{Visualization}}\\

%\\\\\bottomrule
\end{tabular}
\newpage

%
% C6
%
\marginnote[1em]{
% MARGIN TAB
\color{red}
\begin{tabular}{rccc}
\toprule
& \textsc{WoS} & \textsc{Sco} & \textsc{GSc}\\
l. cytowań & 4 & 5 & 5 \\\\
\multicolumn{3}{r}{\color{black}\emph{Szacowany udział}} & \color{black}100\%\\
\multicolumn{3}{r}{\color{black}\emph{Core}} & \color{black}C\\
\multicolumn{3}{r}{\color{black}\emph{l. punktów \textsc{me}i\textsc{n}}} & \color{black} 20\\\\

%\multicolumn{4}{r}{\includegraphics[width=1.5cm]{WoS.png}}

\end{tabular}

% BASE TAB
}\noindent\begin{tabular}{lllll@{}}
\toprule
\color{red}[C6] & \multicolumn{4}{p{28em}}{\fullcite{C6}}  \\
 & \multicolumn{4}{p{30em}}{
 {\footnotesize\textsc{CRediT}}: 
 \mybox{Conceptualization} \mybox{Methodology} \mybox{Software} \mybox{Validation} \mybox{Formal Analysis} \mybox{Investigation} \mybox{Resources} \mybox{Data Curation} \mybox{Writing - Original Draft} \mybox{Writing - Review \& Editing} \mybox{Visualization}}\\

%\\\\\bottomrule
\end{tabular}
\vspace{2.5em}

%
% C7
%
\marginnote[1em]{
% MARGIN TAB
\color{red}
\begin{tabular}{rccc}
\toprule
& \textsc{WoS} & \textsc{Sco} & \textsc{GSc}\\
l. cytowań & 11 & 18 & 28 \\\\
\multicolumn{3}{r}{\color{black}\emph{Szacowany udział}} & \color{black}50\%\\
\multicolumn{3}{r}{\color{black}\emph{Impact Factor}} & \color{black}4.438\\
\multicolumn{3}{r}{\color{black}\emph{l. punktów \textsc{me}i\textsc{n}}} & \color{black} 140\\\\

%\multicolumn{4}{r}{\includegraphics[width=1.5cm]{WoS.png}}

\end{tabular}
% BASE TAB
}\noindent\begin{tabular}{lllll@{}}
\toprule
%\color{red}[C7] & \multicolumn{4}{p{28em}}{\fullcite{C7}}  \\
\color{red}[C7] & \multicolumn{4}{p{28em}}{Paweł Ksieniewicz, Michał Woźniak, Bogusław Cyganek, Andrzej Kasprzak i Krzysztof Walkowiak. \emph{"Data stream classification using active learned neural networks"}. W: \emph{Neurocomputing 353 (2019)}, s. 74-82. \textsc{doi}: \verb|\url{10.1016/j.neucom.2018.05.130}}  \\
 & \multicolumn{4}{p{30em}}{
 {\footnotesize\textsc{CRediT}}: 
 \mybox{Methodology} \mybox{Software} \mybox{Validation} \mybox{Investigation} \mybox{Writing - Original Draft} \mybox{Writing - Review \& Editing} \mybox{Visualization}}\\

%\\\\\bottomrule
\end{tabular}
\vspace{2.5em}


%
% C8
%
\marginnote[1em]{
% MARGIN TAB
\color{red}
\begin{tabular}{rccc}
\toprule
& \textsc{WoS} & \textsc{Sco} & \textsc{GSc}\\
l. cytowań & --- & --- & 10 \\\\
\multicolumn{3}{r}{\color{black}\emph{Szacowany udział}} & \color{black}100\%\\
\multicolumn{3}{r}{\color{black}\emph{Core}} & \color{black}A\\
\multicolumn{3}{r}{\color{black}\emph{l. punktów \textsc{me}i\textsc{n}}} & \color{black} 140\\\\

%\multicolumn{4}{r}{\includegraphics[width=1.5cm]{WoS.png}}

\end{tabular}
% BASE TAB
}\noindent\begin{tabular}{lllll@{}}
\toprule
\color{red}[C8] & \multicolumn{4}{p{28em}}{\fullcite{C8}}  \\
 & \multicolumn{4}{p{30em}}{
 {\footnotesize\textsc{CRediT}}: 
 \mybox{Conceptualization} \mybox{Methodology} \mybox{Software} \mybox{Validation} \mybox{Formal Analysis} \mybox{Investigation} \mybox{Resources} \mybox{Data Curation} \mybox{Writing - Original Draft} \mybox{Writing - Review \& Editing} \mybox{Visualization}}\\

%\\\\\bottomrule
\end{tabular}
\vspace{2.5em}

%
% C9
%
\marginnote[1em]{
% MARGIN TAB
\color{red}
\begin{tabular}{rccc}
\toprule
& \textsc{WoS} & \textsc{Sco} & \textsc{GSc}\\
l. cytowań & 6 & 11 & 13 \\\\
\multicolumn{3}{r}{\color{black}\emph{Szacowany udział}} & \color{black}80\%\\
\multicolumn{3}{r}{\color{black}\emph{Core}} & \color{black}B\\
\multicolumn{3}{r}{\color{black}\emph{l. punktów \textsc{me}i\textsc{n}}} & \color{black} 15\\\\

%\multicolumn{4}{r}{\includegraphics[width=1.5cm]{WoS.png}}

\end{tabular}
% BASE TAB
}\noindent\begin{tabular}{lllll@{}}
\toprule
\color{red}[C9] & \multicolumn{4}{p{28em}}{\fullcite{C9}}  \\
 & \multicolumn{4}{p{30em}}{
 {\footnotesize\textsc{CRediT}}: 
 \mybox{Methodology} \mybox{Software} \mybox{Validation} \mybox{Investigation} \mybox{Writing - Original Draft} \mybox{Writing - Review \& Editing} \mybox{Visualization}}\\

%\\\\\bottomrule
\end{tabular}
\newpage

%
% C10
%
\marginnote[1em]{
% MARGIN TAB
\color{red}
\begin{tabular}{rccc}
\toprule
& \textsc{WoS} & \textsc{Sco} & \textsc{GSc}\\
l. cytowań & 15 & 15 & 18 \\\\
\multicolumn{3}{r}{\color{black}\emph{Szacowany udział}} & \color{black}60\%\\
\multicolumn{3}{r}{\color{black}\emph{Impact Factor}} & \color{black}4.072\\
\multicolumn{3}{r}{\color{black}\emph{l. punktów \textsc{me}i\textsc{n}}} & \color{black} 30\\\\

%\multicolumn{4}{r}{\includegraphics[width=1.5cm]{WoS.png}}

\end{tabular}
% BASE TAB
}\noindent\begin{tabular}{lllll@{}}
\toprule
\color{red}[C10] & \multicolumn{4}{p{28em}}{\fullcite{C10}}  \\
 & \multicolumn{4}{p{30em}}{
 {\footnotesize\textsc{CRediT}}: 
 \mybox{Conceptualization} \mybox{Methodology} \mybox{Software} \mybox{Validation} \mybox{Writing - Original Draft} \mybox{Writing - Review \& Editing} \mybox{Visualization}}\\

%\\\\\bottomrule
\end{tabular}
\vspace{.5em}

%
% C11
%
\marginnote[1em]{
% MARGIN TAB
\color{red}
\begin{tabular}{rccc}
\toprule
& \textsc{WoS} & \textsc{Sco} & \textsc{GSc}\\
l. cytowań & --- & --- & 9 \\\\
\multicolumn{3}{r}{\color{black}\emph{Szacowany udział}} & \color{black}80\%\\
\multicolumn{3}{r}{\color{black}\emph{Core}} & \color{black}A\\
\multicolumn{3}{r}{\color{black}\emph{l. punktów \textsc{me}i\textsc{n}}} & \color{black} 140\\\\

%\multicolumn{4}{r}{\includegraphics[width=1.5cm]{WoS.png}}

\end{tabular}
% BASE TAB
}\noindent\begin{tabular}{lllll@{}}
\toprule
\color{red}[C11] & \multicolumn{4}{p{28em}}{\fullcite{C11}}  \\
 & \multicolumn{4}{p{30em}}{
 {\footnotesize\textsc{CRediT}}: 
 \mybox{Conceptualization} \mybox{Methodology} \mybox{Software} \mybox{Validation} \mybox{Formal Analysis} \mybox{Investigation} \mybox{Resources} \mybox{Data Curation} \mybox{Writing - Original Draft} \mybox{Writing - Review \& Editing} \mybox{Visualization}}\\

\end{tabular}


\begin{fullwidth}
	

\section{INFORMACJA O AKTYWNOŚCI NAUKOWEJ ALBO ARTYSTYCZNEJ}

\subsection{Wykaz opublikowanych monografii naukowych (z zaznaczeniem pozycji niewymienionych w pkt 1.1).}

\begin{itemize}
	\item \textbf{Paweł Ksieniewicz [Red.]}, Mariusz Uchroński [Red.]\\\emph{Selected model based architectures and algorithms for learning, signal processing and optimization}\\Warszawa: Akademicka Oficyna Wydawnicza EXIT, cop. 2021. 144 s. (Problemy Współczesnej Informatyki)
\end{itemize}

\subsection{Wykaz opublikowanych rozdziałów w monografiach naukowych.}

Jestem autorem dwóch rozdziałów w książkach:

\begin{itemize}
	\item D. Sułot, P. Zyblewski, \textbf{P. Ksieniewicz}\\\emph{A novel approach to learning and designing neural networks-based ensemble}\\Selected model based architectures and algorithms for learning, signal processing and optimization / red. Paweł Ksieniewicz, Mariusz Uchroński. Warszawa : Akademicka Oficyna Wydawnicza EXIT, cop. 2021. s. 103-112.
	\item K. Jackowski, D. Jankowski, \textbf{P. Ksieniewicz}, D. Simić, S. Simić, M. Woźniak.\\\emph{Ensemble classifier systems for headache diagnosis.}\\W: Information technologies in biomedicine. Vol. 4 / Ewa Pietka, Jacek Kawa, Wojciech Wieclawek (eds.). Cham [i in.] : Springer, cop. 2014. s. 273-284. (Advances in Intelligent Systems and Computing, ISSN 2194-5357; vol. 284)
\end{itemize}
			
\subsection{Informacja o członkowskie w redakcjach naukowych monografii}

Jestem współredaktorem oraz współautorem rozdziału w książce:

\begin{itemize}
	\item \textbf{Paweł Ksieniewicz [Red.]}, Mariusz Uchroński [Red.]\\\emph{Selected model based architectures and algorithms for learning, signal processing and optimization}\\Warszawa: Akademicka Oficyna Wydawnicza EXIT, cop. 2021. 144 s. (Problemy Współczesnej Informatyki)
\end{itemize}

\subsection{Wykaz opublikowanych artykułów w czasopismach naukowych (z zaznaczeniem pozycji niewymienionych w pkt I.2)}
		
\begin{center}
	\bfseries Informacje dot. liczby punktów MEiN oraz współczynnika IF\\podane na podstawie wskaźników z dnia 27 sierpnia 2022.
\end{center}


\paragraph{2.4a Artykuły opublikowane po uzyskaniu stopnia doktora, zgłoszone w punkcie I.2:}

\begin{itemize}
	\item[1.)] [\textbf{IF: 8.139, PKT: 200}] [C1] \\\fullcite{C1}\vspace{1em}
	\item[2.)] [\textbf{IF:5.019, PKT: 70}] [C2] \\\fullcite{C2}\vspace{1em}
	\item[3.)] [\textbf{IF:5.719, PKT: 140}] [C4] \\\fullcite{C4}\vspace{1em}
	\item[4.)] [\textbf{IF: 4.438, PKT:140}] [C7] \\\fullcite{C7}\vspace{1em}
	\item[5.)] [\textbf{IF: 4.072, PKT: 30}] [C10] \\\fullcite{C10}\vspace{1em}		
\end{itemize}


\paragraph{2.4b Artykuły opublikowane po uzyskaniu stopnia doktora, niezgłoszone w punkcie I.2.:}

\begin{itemize}
	\item[1.)] [\textbf{IF: 5.779, PKT: 140}] [Ksi22a] \\\fullcite{Ksi22a}\vspace{1em}
	\item[2.)] [\textbf{IF: 8.263, PKT: 200}] [Cho21] \\\fullcite{Cho21}\vspace{1em}
	\item[3.)] [\textbf{IF: 2.738, PKT: 100}] [Weg20] \\\fullcite{Weg20}\vspace{1em}
	\item[4.)] [\textbf{IF: 8.139, PKT: 200}] [Ksi21k] \\\fullcite{Ksi21k}\vspace{1em}
	\item[5.)] [\textbf{IF: 3.752, PKT: 100}] [Sul21b] \\\fullcite{Sul21b}\vspace{1em}
	\item[6.)] [\textbf{IF: 8.263, PKT: 200}] [Sta21] \\\fullcite{Sta21}\vspace{1em}
	\item[7.)] [\textbf{IF: 3.476, PKT: 100}] [Woj22] \\\fullcite{Woj22}\vspace{1em}
	\item[8.)] [\textbf{IF: 2.786, PKT: 40}] [Gos22] \\\fullcite{Gos22}\vspace{1em}
	\item[9.)] [\textbf{IF: 4.142, PKT: 140}] [Klin20] \\\fullcite{Klin20}\vspace{1em}
\end{itemize}

\paragraph{2.4c Artykuły opublikowane przed uzyskaniem stopnia doktora}:
\begin{itemize}
	\item[1.)] [Woz16b] \\\fullcite{Woz16b}\vspace{1em}
	\item[2.)] [Ksi17p] \\\fullcite{Ksi17p}\vspace{1em}
	\item[3.)] [Ksi14a] \\\fullcite{Ksi14a}\vspace{1em}
\end{itemize}


\subsection{Wykaz osiągnięć projektowych, konstrukcyjnych, technologicznych (z zaznaczeniem pozycji niewymienionych w pkt I.3)}

---

\subsection{Wykaz publicznych realizacji dzieł artystycznych (z zaznaczeniem pozycji niewymienionych w pkt I.3)}

---

\subsection{Informacja o wystąpieniach na krajowych lub międzynarodowych konferencjach naukowych lub artystycznych, z wyszczególnieniem przedstawionych wykładów na zaproszenie i wykładów plenarnych.}
%\end{fullwidth}
%\begin{fullwidth}

%\newpage	
\paragraph{2.7a Aktywny udział (prezentacja pracy) podczas międzynarodowych konferencji naukowych po uzyskaniu stopnia doktora:}
\begin{itemize}
	\item[1.)]~\\
	
	\begin{tabular}{p{7em}p{32.5em}}
	Konferencja:& 2022 International Joint Conference on Neural Networks (IJCNN)\\
	Miejsce:& Padwa, Włochy\\
	Referat: & \textbf{\fullcite{Kom22}}			
	\end{tabular}
	\item[2.)]~\\
	
	\begin{tabular}{p{7em}p{32.5em}}
	Konferencja:& 2021 International Joint Conference on Neural Networks (IJCNN)\\
	Miejsce: & Shenzen, Chiny\\
	Referat: & \textbf{\fullcite{C3}}					
	\end{tabular}

	\item[3.)]~\\
	
	\begin{tabular}{p{7em}p{32.5em}}
	Konferencja:& ICCS 2020 : 20th International Conference of Computational Science\\
	Miejsce: & Amsterdam, Holandia, 3-5 czerwca 2020\\
	Referat: & \textbf{\fullcite{Ksi20b}}			
	\end{tabular}
	
	\item[4.)]~\\
	
	\begin{tabular}{p{7em}p{32.5em}}
	Konferencja:& 2020 International Joint Conference on Neural Networks (IJCNN)\\
	Miejsce: & Glasgow, Szkocja, 19-24 czerwca 2020\\
	Referat: & \textbf{\fullcite{C5}}			
	\end{tabular}\newpage

	\item[5.)]~\\
	
	\begin{tabular}{p{7em}p{32.5em}}
	Konferencja:& Hybrid Artificial Intelligent Systems : 14th International Conference, HAIS 2019\\
	Miejsce: & León, Hiszpania, 4–6 Września 2019\\
	Referat: & \textbf{\fullcite{C6}}		
	\end{tabular}%\newpage

	\item[6.)]~\\
	
	\begin{tabular}{p{7em}p{32.5em}}
	Konferencja:& Intelligent Data Engineering and Automated Learning - IDEAL 2019 : 20th International Conference\\
	Miejsce: & Manchester, Anglia, 14-16 listopada 2019\\
	Referat: & \textbf{\fullcite{Koz19}}
	\end{tabular}


	\item[7.)]~\\
	
	\begin{tabular}{p{7em}p{32.5em}}
	Konferencja:& Intelligent Data Engineering and Automated Learning - IDEAL 2019 : 20th International Conference\\
	Miejsce: & Manchester, Anglia, 14-16 listopada 2019\\
	Referat: & \textbf{\fullcite{Ksi19f}}	
	\end{tabular}

	\item[8.)]~\\
	
	\begin{tabular}{p{7em}p{32.5em}}
	Konferencja:& Intelligent Data Engineering and Automated Learning - IDEAL 2018 : 19th International Conference\\
	Miejsce: & Madryt, Hiszpania, 21-23 listopada 2018\\
	Referat: & \textbf{\fullcite{C9}}		
	\end{tabular}

	\item[9.)]~\\
	
	\begin{tabular}{p{7em}p{32.5em}}
	Konferencja:& Intelligent Data Engineering and Automated Learning - IDEAL 2018 : 19th International Conference\\
	Miejsce: & Madryt, Hiszpania, 21-23 listopada 2018\\
	Referat: & \textbf{\fullcite{Ksi18c}}
	\end{tabular}\newpage

	\item[10.)]~\\
	
	\begin{tabular}{p{7em}p{32.5em}}
	Konferencja:& Second International Workshop on Learning with Imbalanced Domains: Theory and Applications,\\
	Miejsce: & Dublin, Irlandia, 10 września 2018\\
	Referat: & \textbf{\fullcite{C8}}		
	\end{tabular}	

	\item[11.)]~\\
	
	\begin{tabular}{p{7em}p{32.5em}}
	Konferencja:& First International Workshop on Learning with Imbalanced Domains: Theory and Applications,\\
	Miejsce: & Skopje, Macedonia, 22 września 2017\\
	Referat: & \textbf{\fullcite{C11}}		
	\end{tabular}\vspace{1em}					
\end{itemize}
\end{fullwidth}


\paragraph{2.7b Aktywny udział (prezentacja pracy) podczas międzynarodowych konferencji naukowych przed uzyskaniem stopnia doktora:}
\begin{itemize}
	\item[1.)]~\\
	
	\begin{tabular}{p{7em}p{32.5em}}
	Konferencja:& Image Processing and Communications Challenges 8 : 8th International Conference, IP\&C 2016, \\
	Miejsce: & Bydgoszcz, Polska, wrzesień 2016\\
	Referat: & \textbf{\fullcite{Woz16}}		
	\end{tabular}	
				
	\item[2.)]~\\
	
	\begin{tabular}{p{7em}p{32.5em}}
	Konferencja:& Proceedings of the 9th International Conference on Computer Recognition Systems, CORES 2015\\
	Miejsce: & Wrocław, Polska, 2015\\
	Referat: & \textbf{\fullcite{Ksi16a}}		
	\end{tabular}\newpage
	
	\item[3.)]~\\
	
	\begin{tabular}{p{7em}p{32.5em}}
	Konferencja:& 4th Workshop on Machine Learning in Life Sciences (MLLS), 23 September 2016,\\
	Miejsce: & Riva del Garda, Włochy, 23 września 2016\\
	Referat: & \textbf{\fullcite{Ksi16m}}.	
	\end{tabular}
	
	\item[4.)]~\\
	
	\begin{tabular}{p{7em}p{32.5em}}
	Konferencja:& Computational Collective Intelligence : 7th International Conference, ICCCI 2015\\
	Miejsce: & Madryt, Hiszpania, 21–23 września 2015\\
	Referat: & \textbf{\fullcite{Ksi15a}}		
	\end{tabular}		

	\item[5.)]~\\
	
	\begin{tabular}{p{7em}p{32.5em}}
	Konferencja:& International Joint Conference SOCO'14-CISIS'14-ICEUTE'14, \\
	Miejsce: & Bilbao, Hiszpania, 25-27 czerwca 2014\\
	Referat: & \textbf{\fullcite{Kra14b}}
	\end{tabular}

	\item[6.)]~\\
	
	\begin{tabular}{p{7em}p{32.5em}}
	Konferencja:& Hybrid artificial intelligence systems : 9th international conference, HAIS 2014\\
	Miejsce: & Salamanca, Hiszpania, 11-13 czerwca 2014\\
	Referat: & \textbf{\fullcite{Kra14a}}		
	\end{tabular}

	\item[7.)]~\\
	
	\begin{tabular}{p{7em}p{32.5em}}
	Konferencja:& 4th International Conference, Information Technologies in Biomedicine\\
	Miejsce: & Kamień Śląski, Polska, 2-4 czerwca 2014\\
	Referat: & \textbf{\fullcite{Jac14}}
	\end{tabular}	
\end{itemize}
\newpage
\paragraph{2.7c Wykłady (prezentacje) wygłoszone dla zagranicznych zespołów badawczych:}

\begin{itemize}
	\item[1.)] Wykład na zaproszenie\\
	\emph{Research practices in data stream analysis and imbalanced data classification}\\
	22 czerwca 2020, Amity School of Engineering and Technology, Noida, Indie\\
	
	\item[2.)] Wykład w ramach Elsevier Webinar Machine Learning to combat Fake News and Media Manipulation,\\
	\emph{Using machine learning as the weapon against the disinformation}\\
	20 kwietnia 2021\\
	\url{https://www.workcast.com/register?cpak=7948916184707381}\\
	
	\item[3.)] Keynote podczas sesji specjalnej CLDD w ramach konferencji International Conference on Computational Science,\\
		\emph{Chosen Challenges of Imbalanced Data Stream Classification}\\
		16 czerwca 2021\\
		\url{https://www.iccs-meeting.org/iccs2021/}\\
\end{itemize}

\begin{fullwidth}
	

\subsection{Informacja o udziale w komitetach organizacyjnych i naukowych konferencji krajowych lub międzynarodowych, z podaniem pełnionej funkcji.}

\begin{itemize}
		\item[8a.)] Członek komitetu technicznego podczas międzynarodowych konferencji naukowych po uzyskaniu stopnia doktora:
		
		\begin{itemize}
			\item IEEE International Conference on Omni-layer Intelligent Systems 2022, IEEE COINS 2022
			\item The 12 International Conference on Computer Recognition Systems, CORES 2021,
			\item International Conference on Computational Science 2021, ICCS 2021
			\item 13th International Conference on Computational Intelligence in Security for Information Systems, CISIS 2020,
			\item International Conference on Computational Science 2020, ICCS 2020
			%\item HAIS2020-IGPL (SI-HAIS 2020 IGPL)
			\item 11th International Conference on Image Processing and Communications, IP\&C 2019
			\item The 11 International Conference on Computer Recognition Systems, CORES 2019,
			\item 20th International Conference on Intelligent Data Engineering and Automated Learning, IDEAL 2019
		\end{itemize}
		
		\item[8b.)] Członek komitetu technicznego podczas międzynarodowych konferencji naukowych przed uzyskaniem stopnia doktora
		
		\begin{itemize}
			\item International Conference on Data Mining and Big Data, DMDB 2016
			\item The 9 International Conference on Computer Recognition Systems, CORES 2015,
			\item Hybrid Ensemble Machine Learning for Complex and Dynamic Data, HEMLCDD 2014,
		\end{itemize}


		\item[8c.)] Członek komitetu organizacyjnego specjalnej sesji naukowej po uzyskaniu stopnia doktora
		
		\begin{itemize}
			\item Organizacja sesji specjalnej “Classifier Learning from Difficult Data” na konferencji International Conference on Computational Science, 3–5 czerwca 2020, Amsterdam, Holandia. Zasięg międzynarodowy.
			\item Organizacja sesji specjalnej “Machine Learning Algorithms for Hard Problems“ na konferencji 20th International Conference on Intelligent Data Engineering and Automated Learning (IDEAL), 14–16 listopada 2019, Manchester, Anglia. Zasięg międzynarodowy.
			\item Organizacja sesji specjalnej “Classifier Learning from Difficult Data” na konferencji International Conference on Computational Science, 12–14 czerwca 2019, Faro, Portugalia. Zasięg międzynarodowy.
			\item Organizacja konferencji Polskie Porozumienie na rzecz Rozwoju Sztucznej Inteligencji, 16– 18 października 2019, Wrocław, Polska. Zasięg krajowy.
		\end{itemize}
		
		\item[8d.)] Członek komitetu organizacyjnego specjalnej sesji naukowej przed uzyskaniem stopnia doktora
		
		\begin{itemize}
			\item Organizacja konferencji The 9 International Conference on Computer Recognition Systems, CORES 2015, Wrocław, Polska. Zasięg międzynarodowy
		\end{itemize}
	
	\end{itemize}

\subsection{Informacja o uczestnictwie w pracach zespołów badawczych realizujących projekty finansowane w drodze konkursów krajowych lub zagranicznych, z podziałem na projekty zrealizowane i będące w toku realizacji, oraz z uwzględnieniem informacji o pełnionej funkcji w ramach prac zespołów.}


\begin{itemize}
		\item[9a.)] Projekty w toku, rozpoczęte po uzyskaniu stopnia doktora:	
		
		
\item[1.)]~\\
\begin{tabular}{p{10em}|p{20em}}
Tytuł &	\textbf{System Wykrywania Dezinformacji Metodami Sztucznej Inteligencji}\\
Źródło finansowania & NCBiR\\
Budżet & 8 657 006 zł\\
Okres realizacji & 2021-12-02 – 2024-04-01\\
Partnerzy & Matic S.A., Politechnika Bydgoska\\
Rola w projekcie & \textbf{Kierownik B+R}.
\end{tabular}

\item[2.)]~\\
\begin{tabular}{p{10em}|p{20em}}
Tytuł &	\textbf{Incat FaaS AI - Opracowanie platformy bezpieczeństwa operacyjnego podmiotów finansowych w oparciu o zaawansowane mechanizmy uczenia maszynowego}\\
Źródło finansowania & NCBiR\\
Budżet & 8 267 875 zł\\
Okres realizacji & 2020-04-01 -- 2023-03-31\\
Partnerzy & INCAT Spółka z ograniczoną odpowiedzialnością\\
Rola w projekcie & Ekspert AI
\end{tabular}

\item[3.)]~\\
\begin{tabular}{p{10em}|p{20em}}
Tytuł &	\textbf{Math Solution innowacyjna platforma wspomagająca uczniów i korepetytorów w procesie nauczania indywidualnego lub wspólnego w oparciu o zaawansowane algorytmy przetwarzania obrazu i uczenia maszynowego w zakresie matematyki i innych przedmiotów ścisłych}\\
Źródło finansowania & NCBiR\\
Budżet & 9 162 812,50 zł\\
Okres realizacji & 2021-11-01 -- 2023-10-01\\
Partnerzy & Sirius Education Sp. z o.o.\\
Rola w projekcie & Ekspert AI
\end{tabular}

%\item[4.)]~\\
%\begin{tabular}{p{10em}|p{20em}}
%Tytuł &	---\\
%Źródło finansowania & Polska Agencja Rozwoju Przedsiębiorczości\\
%Budżet & ---\\
%Okres realizacji & ---\\
%Partnerzy & Hinter \\
%Rola w projekcie & Ekspert AI
%\end{tabular}

\item[5.)]~\\
\begin{tabular}{p{10em}|p{20em}}
Tytuł &	\textbf{Optymalizacja kognitywnych sieci optycznych}\\
Źródło finansowania & NCN\\
Budżet & 557 200 zł\\
Okres realizacji & 2018-09-03 – 2022-09-02\\
Rola w projekcie & Wykonawca
\end{tabular}

\item[6.)]~\\
\begin{tabular}{p{10em}|p{20em}}
Tytuł &	\textbf{Optymalizacja szkieletowych sieci optycznych z wykorzystaniem narzędzi modelowania predykcji ruchu sieciowego}\\
Źródło finansowania & NCN\\
Budżet & 218 400 zł\\
Okres realizacji & 2019-10-01 – 2022-09-30\\
Rola w projekcie & Wykonawca
\end{tabular}
		
		\item[9b.)] Projekty zakończone, realizowane po uzyskaniu stopnia doktora:

\item[1.)]~\\
\begin{tabular}{p{10em}|p{20em}}
Tytuł &	\textbf{Algorytmy klasyfikacji niezbalansowanych strumieni danych}\\
Źródło finansowania & NCN\\
Budżet & 613 920 zł\\
Okres realizacji & 2018-09-04 – 2021-09-30\\
Rola w projekcie & Wykonawca
\end{tabular}

\item[2.)]~\\
\begin{tabular}{p{10em}|p{20em}}
Tytuł &	\textbf{Metody klasyfikacji wieloklasowej danych niezbalansowanych}\\
Źródło finansowania & NCN\\
Budżet & 440 044 zł\\
Okres realizacji & 2016-07-22— 2020-01-21\\
Rola w projekcie & Wykonawca
\end{tabular}

\item[3.)]~\\
\begin{tabular}{p{10em}|p{20em}}
Tytuł &	\textbf{Integration of base classifiers in geometrical space}\\
Źródło finansowania & NCN\\
Budżet & 497 980 zł\\
Okres realizacji & 2018-01-25 – 2021-01-24\\
Rola w projekcie & Wykonawca
\end{tabular}

\item[4.)]~\\
\begin{tabular}{p{10em}|p{20em}}
Tytuł &	\textbf{European Union's Horizon 2020 / SocialTruth}\\
Źródło finansowania & EU H2020\\
Budżet & 13 665 800 zł\\
Okres realizacji & 2019-05-16 – 2021-11-30\\
Rola w projekcie & Researcher/Expert
\end{tabular}
\newpage
		\item[9c.)] Projekty zakończone, realizowane przed uzyskaniem stopnia doktora:

\item[1.)]~\\
\begin{tabular}{p{10em}|p{20em}}
Tytuł &	\textbf{Złożone metody klasyfikacji danych strumieniowych wykorzystujące paradygmaty uczenia nienadzorowanego i aktywnego}\\
Źródło finansowania & NCN\\
Budżet & 669 292 zł\\
Okres realizacji & 2014-03-12 – 2017-03-11\\
Rola w projekcie & Wykonawca
\end{tabular}
	\end{itemize}



\subsection{Członkostwo w międzynarodowych lub krajowych organizacjach i towarzystwach naukowych wraz z informacją o pełnionych funkcjach}

---

\subsection{Informacja o odbytych stażach w instytucjach naukowych lub artystycznych, w tym zagranicznych, z podaniem miejsca, terminu, czasu trwania stażu i jego charakteru.}

\paragraph{11a.) Staże zrealizowane po uzyskaniu stopnia doktora:}

\end{fullwidth}
\begin{itemize}
			\item[1.)]~\\
			\begin{tabular}{p{7em}p{20em}}
				Jednostka & \textbf{Universidad del Pais Vasco}\\
				& San~Sebastian,~Hiszpania\\
				Termin stażu & 15-08-2019 -- 30-08-2019\\
				Tematyka & \emph{Redakcja wniosków projektowych i przetwarzanie sygnałów cyfrowych}\\
				Charakter stażu & Celem stażu była wymiana wiedzy w zakresie metodyki prowadzenia i redakcji wniosków projektowych oraz wspólne badania z zakresu przetwarzania cyfrowych sygnałów wielowymiarowych. Podjęte prace badawcze, przerwane ze względu na epidemię COVID-19, zostały ostatnio wznowione, ale nie przyniosły jeszcze efektów w postaci publikacji artykułów naukowych. \\
				 %Efekty:& \vspace{.5em} Pawel Ksieniewicz et al. “SWAROG – fake news classification for the local context”. In\\
				 %&\vspace{.5em} Michał Choraś et al. “SocialTruth - content verification for the digital society”. In
			\end{tabular}\newpage
			\item[2.)]~\\
			\begin{tabular}{p{7em}p{20em}}
				Jednostka & \textbf{Virginia Commonwealth University}\\
				& Department of Computer Science\\
				& School of Engineerig\\
				& Richmond,~VA,~USA\\
				Termin stażu & 16-10-2019 -- 26-10-2019\\
				Tematyka & \emph{Przetwarzanie trudnych strumieni danych}\\
				Charakter stażu & Celem stażu naukowego było wypracowanie nowych koncepcji z zakresu przetwarzania trudnych strumieni danych. Głównym efektem stażu stało się opracowanie strategii dywersyfikacji komitetów podprzestrzennych przez losowanie ich cech dominujących z rozkładów niejednostajnych, wyznaczonych przez analizę strumienia. Efektem dodatkowym są prowadzone wciąż badania nad rozszerzeniem tej strategii do zagadnienia wyjaśnialnych modeli rozpoznawania.\\
				
%				 Efekty:& \vspace{.5em} Weronika Wegier and Pawel Ksieniewicz. “Application of Imbalanced Data Classification Quality Metrics as Weighting Methods of the Ensemble Data Stream Classification Algorithms”. In: Entropy 22.8 (July 2020), p. 849\\
%				 & \vspace{.5em}Dominika Sułot, Paweł Zyblewski, and Paweł Ksieniewicz. “Analysis of Variance Application in the Construction of Classifier Ensemble Based on Optimal Feature Subset for the Task of Supporting Glaucoma Diagnosis”. In: Computational Science – ICCS 2021. Springer International Publishing, 2021, pp. 109–117.
				
			\end{tabular}
		\end{itemize}

\paragraph{11a.) Staże zrealizowane przed uzyskaniem stopnia doktora:}

\begin{itemize}
			\item[1.)]~\\
			\begin{tabular}{p{7em}p{20em}}
				Jednostka badawcza & \textbf{Universidad del Pais Vasco}\hspace{5em} San~Sebastian,~Hiszpania\\
				Termin stażu & 06-02-2016 -- 21-02-2016\\
				Tematyka & Klasyfikacja obrazów nadwidmowych\\
				Charakter stażu & W ramach stażu uczestniczyłem w bieżących pracach zespołu kierowanego przez prof. Manuela Granę, zdobywając cenne doświadczenia z zakresu metodyki pracy naukowej. Wyjazd stanowił kontynuację współpracy podjętej już w roku 2014. Ponadto, uczestniczyłem w seminariach poświęconych przetwarzaniu danych nadwidmowych -- stanowiących kluczowy obszar pracy doktorskiej oraz sformalizowałem metodę kluczową dla realizacji pracy doktorskiej.\\
%				 Efekty:& \vspace{.5em} Paweł Ksieniewicz, Manuel Graña, and Michał Woźniak. “Blurred Labeling Segmentation Algorithm for Hyperspectral Images”. In: Computational Col- lective Intelligence. Springer International Publishing, 2015, pp. 578–587.\\
%				 &\vspace{.5em}Pawel Ksieniewicz and Michał Woźniak. “Artificial Photoreceptors for En- semble Classification of Hyperspectral Images”. In: Advances in Intelligent Systems and Computing. Springer International Publishing, 2016, pp. 471– 479.\\
%				& \vspace{.5em} Paweł Ksieniewicz, Manuel Graña, and Michał Woźniak. “Blurred Labeling Segmentation Algorithm for Hyperspectral Images”. In: Computational Col- lective Intelligence. Springer International Publishing, 2015, pp. 578–587.\\
%				& \vspace{.5em} Bartosz Krawczyk, Paweł Ksieniewicz, and Michał Woźniak. “Hyperspectral Image Analysis Based on Color Channels and Ensemble Classifier”. In: Lecture Notes in Computer Science. Springer International Publishing, 2014, pp. 274–284.\vspace{1em}
				\end{tabular}
				
\end{itemize}

\begin{fullwidth}
	
\subsection{Członkostwo w komitetach redakcyjnych i radach naukowych czasopism wraz z informacją o pełnionych funkcjach}

---

\newpage

\subsection{Informacja o recenzowanych pracach naukowych lub artystycznych, w szczególności publikowanych w czasopismach międzynarodowych.}

Wykonywałem recenzje prac dla następujących czasopism naukowych:

\begin{itemize}
	\item[] \begin{tabular}{p{25em}p{5em}}
\textsc{Czasopismo} & \textsc{Impact Factor}\\\midrule
	Machine Learning & 2.940\\ 
	Pattern Analysis and Applications & 1.307 \\
	Applied Soft Computing & 8.263 \\ 
	%Machine Learning & 2.940\\ 
	Pattern Recognition & 7.74\\ 
	Journal of Computational Science & 3.976\\ 
	IEEE’s Geoscience and Remote Sensing Letters & 3.966\\
	IEEE Access & 3.367\\
	Symmetry & 2.713\\
	Entropy & 2.524
\end{tabular}\vspace{1em}

\subsection{Informacja o uczestnictwie w programach europejskich lub innych programach międzynarodowych}
\end{fullwidth}

\begin{itemize}
	\item[] \begin{tabular}{p{10em}|p{20em}}
Tytuł &	\textbf{European Union's Horizon 2020 / SocialTruth}\\
Źródło finansowania & EU H2020\\
Budżet & 13 665 800 zł\\
Okres realizacji & 2019-05-16 – 2021-11-30\\
Rola w projekcie & Researcher/Expert
\end{tabular}\vspace{1em}


	\item[] \begin{tabular}{p{10em}|p{20em}}
Tytuł &	\textbf{ENGINE – European research centre of Network intelliGence for IN-novation Enhacement}\\
Źródło finansowania & EC: Coordination and support actions (Supporting Action) Work programme topics addressed: Capacities Work Programme: Research Potential\\
Budżet & 4 731 164 EUR\\
Okres realizacji & 2013-06-01 – 2016-12-31\\
Rola w projekcie & Wykonawca
\end{tabular}\vspace{1em}

\end{itemize}

\begin{fullwidth}
	
\subsection{Informacja o udziale w zespołach badawczych, realizujących projekty inne niż określone w pkt. II.9.}

---

\subsection{Informacja o uczestnictwie w zespołach oceniających wnioski o finansowanie badań, wnioski o przyznanie nagród naukowych, wnioski w innych konkursach mających charakter naukowy lub dydaktyczny}

---


\section{INFORMACJA O WSPÓŁPRACY Z OTOCZENIEM SPOŁECZNYM I GOSPODARCZYM}

\subsection{Wykaz dorobku technologicznego}

---


\subsection{Informacja o współpracy z sektorem gospodarczym}


\begin{itemize}
	\item W roku 2016 współorganizowałem wraz z pracownikami \emph{Credit Suisse Group AG} maraton programistyczny JellyPizzaHack dla studentów Politechniki Wrocławskiej.
	\item uczestnictwo w roli eksperta w debacie “\emph{Szczepionka na kłamstwo}.” organizowanej przez Fundację na rzecz Nauki Polskiej w ramach cyklu „\emph{Ufajmy nauce}” (21.04.2022r.),
	\item prelekcja “\emph{Wykorzystanie sztucznej inteligencji w walce z dezinformacją}” w ramach seminarium "\emph{Sztuczna inteligencja w rozwoju miast i obszarów metropolitarnych}" organizowanego przez Wrocławskie Centrum Akademickie pod patronatem World Urban Forum (9.03.2022r.),
	\item wykład w ramach seminarium “\emph{Machine Learning to Combat Fake News and Media Manipulation}” organizowanego przez Elsevier w ramach cyklu webinarów (20.04.2022r.).
	\item wywiady radiowe dotyczące detekcji źródeł dezinformacji.
\end{itemize}


\end{fullwidth}
\begin{fullwidth}
	
\subsection{Uzyskane prawa własności przemysłowej, w tym uzyskane patenty krajowe lub międzynarodowe}
---

\subsection{Informacja o wdrożonych technologiach}
---

\subsection{Informacja o wykonanych ekspertyzach lub innych opracowaniach wykonanych na zamówienie instytucji publicznych lub przedsiębiorców}

\end{fullwidth}
Opracowałem wniosek projektowy o finansowanie prac badawczo-rozwojowych do funduszu \emph{EVIGAlfa}\sidenote{\url{https://evigalfa.pl}} dla startupu \emph{Hinter.ai}\sidenote{\url{https://hinter.ai}}. W roku 2022, zgodnie z opracowaną przeze mnie agendą, zrealizowałem prace badawcze w ramach tego projektu zakończone wdrożeniem modułu automatycznej analizy danych. Efektem działań jest system przygotowujący dla menedżerów przedsiębiorstw listę zalecanych działań, których wdrożenie pomoże usprawnić funkcjonowanie organizacji, który w czerwcu 2022 otrzymał główną nagrodę w konkursie \emph{Polskiej Agencji Rozwoju Przedsiębiorczości}\sidenote[][-2em]{\href{https://pap-mediaroom.pl/biznes-i-finanse/najlepsze-programy-rozwoju-inicjatyw-pracowniczych-parp-wylonil-zwyciezcow}{https://pap-mediaroom.pl/biznes-i-finanse/najlepsze-programy-rozwoju}} dla przedsiębiorców funkcjonujących na rynku nie dłużej niż 3 lata.

\begin{fullwidth}

\subsection{Informacja o udziale w zespołach eksperckich lub konkursowych}
---

\subsection{Informacja o projektach artystycznych realizowanych ze środowiskami pozaartystycznymi.}
---

\newpage


\section{INFORMACJE NAUKOMETRYCZNE}


\begin{center}
	\bfseries Informacje dot. liczby punktów MEiN, współczynnika IF oraz liczby cytowań podane na podstawie wskaźników z dnia 27 sierpnia 2022.
\end{center}

\end{fullwidth}
		
		\begin{itemize}
		
			\item[1.] Informacja o punktacji Impact Factor (w dziedzinach i dyscyplinach w których parametr ten jest powszechnie używany jako wskaźnik naukometryczny).\vspace{-1em}	
			\begin{center}
				\begin{tabular}{p{18em}|p{7em}|p{7em}}
			& Liczba prac z IF & Suma IF\\\midrule
			Ogółem & 16 & 75,743\\%\hline
			Po uzyskaniu stopnia doktora & 15 & 75,309\\%\hline
			Przed uzyskaniem stopnia doktora & 1 & 0,434
				
				\end{tabular}	
			\end{center}
			
			\item[2.] Informacja o liczbie cytowań publikacji wnioskodawcy, z oddzielnym uwzględnieniem autocytowań.\vspace{-1em}
			\begin{center}
				\begin{tabular}{p{18em}|p{7em}|p{7em}}
			& Liczba wszystkich cytowań & Liczba cytowań bez autocytowań\\\midrule
			Google Scholar & 377 & Brak danych\\%\hline
			Web of Science & 157 & 134\\%\hline
			Scopus & 276 & 240
				
				\end{tabular}	
			\end{center}

			\item[3.] Informacja o posiadanym indeksie Hirscha.\vspace{-1em}
			\begin{center}
				\begin{tabular}{p{18em}|p{14em}}
			& h-indeks\\\midrule
			Google Scholar & 13\\%\hline
			Web of Science & 8\\%\hline
			Scopus & 10
				
				\end{tabular}	
			\end{center}
			
			\item[4.] Informacja o liczbie punktów MEiN.\vspace{-1em}			
			\begin{center}
				\begin{tabular}{p{18em}|p{7em}|p{7em}}
			& Liczba prac z listy MEiN & Suma punktów MEiN\\\midrule
			Ogółem & 49 & 3 995\\%\hline
			Po uzyskaniu stopnia doktora & 39 & 3 709\\%\hline
			Przed uzyskaniem stopnia doktora & 10 & 286
				\end{tabular}	
			\end{center}
			
			
		\end{itemize}

\vfill\noindent\hspace{.75\textwidth}\begin{minipage}{15em}
\begin{center}
	\hbox to 5cm{\leaders\hbox to 3pt{\hss . \hss}\hfil}

	(podpis wnioskodawcy)
\end{center}
\end{minipage}