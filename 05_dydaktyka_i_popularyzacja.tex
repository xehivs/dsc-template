
\section{Informacja o osiągnięciach dydaktycznych, organizacyjnych oraz popularyzujących naukę lub sztukę}

Podczas dziewięciu lat pracy naukowo-dydaktycznej prowadziłem dziewiętnaście kursów dla studentów kierunków \emph{Informatyka}, \emph{Informatyka Techniczna} i \emph{Teleinformatyka}, w siedmiu z nich będąc głównym wykładowcą i autorem materiałów dydaktycznych: 

\begin{enumerate}
	\item Metody sztucznej inteligencji%\marginnote[-.5em]{\emph{(laboratorium, \textbf{wykład}, projekt)}},\vspace{1em}
	\item Methods of Computational Intelligence and Decision Making,%\marginnote[-.5em]{\emph{(laboratorium, projekt, \textbf{wykład})}}\vspace{1em}
	\item Obrazowanie biomedyczne,%\marginnote[-.5em]{\emph{(laboratorium, seminarium, \textbf{wykład})}},\vspace{1em}
	\item Przetwarzanie sygnałów wielowymiarowych,%\marginnote[-.5em]{\emph{(\textbf{wykład}, laboratorium)}},\vspace{1em}
	\item Projektowanie systemów internetowych i mobilnych,%\marginnote[-.5em]{\emph{(projekt, seminarium, \textbf{wykład})}} \vspace{1em}
	\item Projektowanie telemedycznych systemów internetowych i mobilnych,%\marginnote{\emph{(laboratorium, \textbf{wykład})}}\vspace{1em}
	\item Aplikacje mobilne.%\marginnote{\emph{(laboratorium, \textbf{wykład})}}.
\end{enumerate}

Angażuję się również w opracowywanie materiałów dydaktycznych dla studentów, uczniów oraz nauczycieli, promocję nauki przez studenckie warsztaty naukowe i współpracę z kołami naukowymi oraz promocję uczelni przez organizację \emph{hackathonów}:

\begin{itemize}
	\item Przygotowałem materiały multimedialne do nauki sztucznej inteligencji oraz serię materiałów wideo dla projektu \emph{Centrum Mistrzostwa Informatycznego}\sidenote[][-2em]{\emph{Centrum Mistrzostwa Informatycznego}\\\noindent\url{https://cmi.edu.pl}}. 
	\item Współorganizowałem kilka edycji studenckich warsztatów naukowych \emph{International Students Workshop}\sidenote[][-1.5em]{\emph{International Students Workshop}\\\noindent\url{http://sisk.kssk.pwr.edu.pl/isw/}} (2015-2019).
	\item Opiekuję się pracami badawczymi studentów w ramach działającego przy \emph{Katedrze Systemów i Sieci Komputerowych}, \emph{Koła Naukowego Systemów i Sieci Komputerowych} oraz założonego w zeszłym roku \emph{Koła Uczenia Maszyn}. 
	\item Byłem pomysłodawcą i jednym z głównych organizatorów \emph{hackathonu} \emph{JellyPizzaHack}\sidenote[][-1.5em]{\emph{JellyPizzaHack}\\\noindent\url{https://invest-in-wroclaw.pl/piwnica-dynamicznie-alokowana}}, zorganizowanego we współpracy z \emph{Credit Suisse} (16.12.2016).
\end{itemize}

Aktywnie uczestniczę także w opracowywaniu kart przedmiotów, będąc opiekunem  czterech kursów:

\begin{itemize}
	\item Projektowanie systemów informatyki medycznej,% (INES00123),
	\item Przetwarzanie sygnałów wielowymiarowych,% (INEU00127),
	\item Metody przetwarzania języka naturalnego oraz wyszukiwanie,% (INEU00129),
	\item Uczenie Maszyn.% (INEA00236).
\end{itemize}

Ponadto, od pięciu lat pełnię rolę sekretarza komisji dyplomowej specjalności \emph{Advanced Informatics and Control}, prowadzonej w języku angielskim. W tym czasie byłem także promotorem 43 prac magisterskich i 37 prac inżynierskich. Sześcioro z moich dyplomantów aktualnie realizuje swoje prace doktorskie na \emph{Politechnice Wrocławskiej}, a z pięciorgiem z nich miałem przyjemność pracować przy ich pierwszych publikacjach konferencyjnych lub publikacjach w czasopismach. Poniżej prezentuję wybraną tematykę prowadzonych prac magisterskich na przykładzie obecnych doktorantów, odnosząc się także do wspólnych prac badawczych:

\begin{itemize}
	\item Prior Probability Estimation in Dynamically Imbalanced Data Streams, 2022,~\citeC[-3em]{C3}\\\emph{mgr inż. Joanna Komorniczak}\vspace{7em}
	\item Feature extraction using n-gram methods for the purpose of ensemble classification of disinformation sources, 2021,~\citeM[-1.5em]{Bor22a}\\\emph{mgr inż. Weronika Borek-Marciniec -- pełnię rolę promotora pomocniczego}\vspace{2em}
	\item Wykorzystanie uczenia zespołowego w klasyfikacji binarnej niezbalansowanych strumieni danych, 2020,~\citeM[-1.5em]{Weg20}\\\emph{mgr inż. Weronika Węgier}\vspace{3em}
	\item Analiza efektywności zastosowania sieci rekurencyjnych w zadaniu klasyfikacji, 2019,~\citeM[-1.5em]{Koz19}\\\emph{mgr inż. Jędrzej Kozal},\vspace{3em}
	\item Analiza efektywności odmian algorytmu \textsc{smote} w balansowaniu strumieni danych, 2019,~\citeM[-1.5em]{Gul19}\\\emph{mgr inż. Bogdan Gulowaty},\vspace{3em}
	\item Zadanie uczenia nienadzorowanego w kontekście obrazowania nadwidmowego, 2018,~\citeM[-1.5em]{Sul21}\\\emph{mgr inż. Dominika Sułot}\vspace{3.5em}
\end{itemize}

Aktywnie udzielam się również w organizacji konferencji i warsztatów naukowych poświęconych zagadnieniom klasyfikacj danych trudnych, wśród których chciałbym wymienić:

\begin{itemize}
    \item \marginnote[1em]{\url{https://www.iccs-meeting.org/iccs2021/}}Organizację sesji specjalnej “\emph{Classifier Learning from Difficult Data}” na konferencji \emph{International Conference on Computational Science} (\textsc{iccs})\\
    	16–18 czerwca 2021, Kraków, Polska. Zasięg międzynarodowy.
    \item \marginnote[1em]{\url{https://www.iccs-meeting.org/iccs2020/}}Organizację sesji specjalnej “\emph{Classifier Learning from Difficult Data}” na konferencji \emph{International Conference on Computational Science} (\textsc{iccs})\\
    	3–5 czerwca 2020, Amsterdam, Holandia. Zasięg międzynarodowy.
    \item \marginnote[1em]{\url{http://www.confercare.manchester.ac.uk/events/ideal2019/sessions/}} Organizację sesji specjalnej “\emph{Machine Learning Algorithms for Hard Problems}“ na konferencji \emph{20th International Conference on Intelligent Data Engineering and Automated Learning} (\textsc{ideal})\\
    	14–16 listopada 2019, Manchester, Anglia. Zasięg międzynarodowy.
    \item \marginnote[1em]{\url{https://www.iccs-meeting.org/iccs2019/}}Organizację sesji specjalnej “\emph{Classifier Learning from Difficult Data}” na konferencji \emph{International Conference on Computational Science} (\textsc{iccs})\\
    	12–14 czerwca 2019, Faro, Portugalia. Zasięg międzynarodowy.
    \item \marginnote[1em]{\url{http://pp-rai.pwr.edu.pl/}}Organizację konferencji \emph{Polskie Porozumienie na rzecz Rozwoju Sztucznej Inteligencji},\\
    	16–18 października 2019, Wrocław, Polska. Zasięg krajowy.
    \item \marginnote[1em]{\url{http://cores.pwr.wroc.pl}}Organizację konferencji \emph{The 9 International Conference on Computer Recognition Systems} \textsc{cores},\\
    	Wrocław, Polska. Zasięg międzynarodowy
\end{itemize}

Wielokrotnie udzielałem się także w działaniach mających na celu zwiększenie świadomości społecznej z zakresu ryzyka związanego z~dezinformacją i~potencjału metod sztucznej inteligencji w~walce z~nią. Należy tu wymienić:

\begin{itemize}
    \item uczestnictwo w roli eksperta w debacie “\emph{Szczepionka na kłamstwo.}” organizowanej przez \emph{Fundację na rzecz Nauki Polskiej} w ramach cyklu „\emph{Ufajmy nauce}” (21.04.2022r.),
    \item[] \url{https://www.fnp.org.pl/debata-ekspertow-szczepionka-na-klamstwo}
    \item[] \url{https://www.youtube.com/watch?v=O75LsK0M4D4} 
    \item wygłoszenie prelekcji pod tytułem 
    \item[] \emph{Wykorzystanie sztucznej inteligencji w walce z dezinformacją} 
    \item[] w ramach seminarium "\emph{Sztuczna inteligencja w rozwoju miast i obszarów metropolitarnych}" organizowanego przez \emph{Wrocławskie Centrum Akademickie} pod patronatem \emph{World Urban Forum} (9.03.2022r.),
    \item[] \url{https://metropolie.pl/artykul/seminarium-sztuczna-inteligencja-dla-} \url{rozwoju-miast-i-obszarow-metropolitalnych}
    \item[] \url{https://vimeo.com/685793217}
    \item wykład w ramach seminarium “\emph{Machine Learning to Combat Fake News and Media Manipulation}” organizowanego przez Elsevier w ramach cyklu webinarów (20.04.2022r.).
    \item[] \url{https://www.workcast.com/register?cpak=7948916184707381}
    \item wywiady radiowe dotyczące detekcji źródeł dezinformacji:
    \begin{itemize}
	    \item[] \emph{Radio RAM\\Politechnika Wrocławska pracuje nad systemem wykrywającym fake newsy}
	    \item[] \url{https://www.radiowroclaw.pl/articles/view/110758/Politechnika-} \url{Wroclawska-pracuje-nad-systemem-wykrywajacym-fake-newsy} 
	\end{itemize}
\end{itemize}

W ostatnich latach przeprowadziłem również trzy wykłady na zaproszenie, dotyczące tematyki klasyfikacji danych trudnych oraz detekcji źródeł dezinformacji:

\begin{itemize}
    \item Keynote podczas sesji specjalnej CLDD w ramach konferencji International Conference on Computational Science.\\\vspace{.5em}
    {\large Chosen Challenges of Imbalanced Data Stream Classification}\\
    16 czerwca 2021\\
    \url{https://www.iccs-meeting.org/iccs2021/}
    \item Wykład na zaproszenie:\\\vspace{.5em}
    {\large Research practices in data stream analysis and imbalanced data classification.}\\
    22 czerwca 2020, Amity School of Engineering and Technology, Noida, Indie,
    \item Wykład w ramach Elsevier Webinar Machine Learning to combat Fake News and Media Manipulation.\\\vspace{.5em}
    {\large Using machine learning as the weapon against the disinformation}\\
    20 kwietnia 2021\\
    \url{https://www.workcast.com/register?cpak=7948916184707381}
\end{itemize}

\section{Dane bibliometryczne}


\subsection{Prace autorstwa Pawła Ksieniewicza spoza cyklu}
{\large
\begin{itemize}
	\item[[Ksi22a]] \fullcite{Ksi22a}
	\item[[Ksi22y]] \fullcite{Ksi22y}
	\item[[Ksi22z]] \fullcite{Ksi22z}
	\item[[Bor22a]] \fullcite{Bor22a}
	\item[[Gos22]] \fullcite{Gos22}
	\item[[Kom22]] \fullcite{Kom22}
	\item[[Kom22c]] \fullcite{Kom22c}
	\item[[Kom22b]] \fullcite{Kom22b}
	\item[[Kom22p]] \fullcite{Kom22p}
	\item[[Woj22]] \fullcite{Woj22}
	
	\item[[Ksi21k]] \fullcite{Ksi21k}
	\item[[Cho21]] \fullcite{Cho21}
	\item[[Sta21]] \fullcite{Sta21}
	\item[[Sul21]] \fullcite{Sul21}
	\item[[Sul21b]] \fullcite{Sul21b}
	
	\item[[Ksi20s]] \fullcite{Ksi20s}
	\item[[Ksi20b]] \fullcite{Ksi20b}
	\item[[Ksi20e]] \fullcite{Ksi20e}
	\item[[Klin20]] \fullcite{Klin20}
	\item[[Kul20]] \fullcite{Kul20}
	\item[[Weg20]] \fullcite{Weg20}
	\item[[Zyb20a]] \fullcite{Zyb20a}
	
	\item[[Ksi19f]] \fullcite{Ksi19f}
	\item[[Gul19]] \fullcite{Gul19}
	\item[[Kli19]] \fullcite{Kli19}
	\item[[Koz19]] \fullcite{Koz19}
	\item[[Zyb19]] \fullcite{Zyb19}
	
	\item[[Ksi18e]] \fullcite{Ksi18e}
	\item[[Ksi18c]] \fullcite{Ksi18c}
	\item[[Lap18]] \fullcite{Lap18}
	
	\item[[Ksi17p]] \fullcite{Ksi17p}
\end{itemize}}

\subsection{Prace autorstwa Pawła Ksieniewicza przed uzyskaniem stopnia naukowego doktora inżyniera}
{\large
\begin{itemize}
	\item[[Ksi16a]] \fullcite{Ksi16a}
	\item[[Ksi16b]] \fullcite{Ksi16b}
	\item[[Ksi16m]] \fullcite{Ksi16m}
	\item[[Woz16]] \fullcite{Woz16}
	\item[[Woz16b]] \fullcite{Woz16b}
	\item[[Woz16c]] \fullcite{Woz16c}
	\item[[Woz16d]] \fullcite{Woz16d}

	\item[[Ksi15a]] \fullcite{Ksi15a}

	\item[[Ksi14a]] \fullcite{Ksi14a}
	\item[[Jac14]] \fullcite{Jac14}
	\item[[Kra14a]] \fullcite{Kra14a}
	\item[[Kra14b]] \fullcite{Kra14b}
\end{itemize}}


\vfill\noindent\hspace{.75\textwidth}\begin{minipage}{15em}
\begin{center}
	\hbox to 5cm{\leaders\hbox to 3pt{\hss . \hss}\hfil}

	(podpis wnioskodawcy)
\end{center}
\end{minipage}
