
\begin{fullwidth}
\subsection{Wykaz publikacji wchodzących w skład cyklu}


Zamieszczony poniżej ciąg publikacji ułożony jest w odwrotnej kolejności chronologicznej, stanowiąc listę jedenastu artykułów opublikowanych w latach 2017--2022. Wszystkie podane przy nich wartości bibliometryczne oddają \textbf{stan na dzień 27. sierpnia 2022 r.} zgodnie z bazami publikacji naukowych:
\end{fullwidth}

\begin{itemize}
	\item[\textsc{WoS}] \emph{Web of Science}%\marginnote[0em]{\noindent\includegraphics[width=1.5cm]{WoS.png}}
	%\begin{flushright}
	
	\url{https://www.webofscience.com/wos/author/record/1886494}	
	%\end{flushright}
	
	\item[\textsc{Sco}] \emph{Scopus}%\marginnote[0em]{\noindent\includegraphics[width=1.5cm]{Scopus.png}}
	%\begin{flushright}
	
	\url{https://www.scopus.com/authid/detail.uri?authorId=56206176100}
	%\end{flushright}
	
	\item[\textsc{GSc}] \emph{Google Scholar}%\marginnote[0em]{\noindent\includegraphics[width=1.5cm]{GSc.png}}
	%\begin{flushright}
	
	\url{https://scholar.google.com/citations?user=YSM30D8AAAAJ}
	%\end{flushright}\vspace{1em}
	\vfill
	
	\item Wpisy bibliograficzne dla artykułów w czasopismach zostały uzupełnione o~informację o wartości czynnika \emph{Impact Factor}\sidenote[][-1em]{Journal Citation Reports --- \url{https://clarivate.com/webofsciencegroup/solutions/journal-citation-reports}} wydawnictwa z~roku publikacji.
	\item Dla artykułów konferencyjnych podano informację o~poziomie konferencji według rankingu \textsc{core}\footnote{CORE Rankings Portal --- \url{https://www.core.edu.au/conference-portal}} w~roku publikacji. 
	\item Każdy wpis zawiera także informację o~liczbie punktów \emph{Ministerstwa Edukacji i Nauki} (\textsc{me}i\textsc{n}) w roku publikacji.
	\item Do każdej pozycji ze spisu publikacji wchodzących w skład cyklu została dodana również informacja o moim wkładzie autorskim, zgodnie z wytycznymi wzorca \textsc{CRediT} (\emph{Contributor Roles Taxonomy})\sidenote[][-3em]{\textsc{CRediT} author statement -- \url{https://www.elsevier.com/authors/policies-and-guidelines/credit-author-statement}}. Określenia te pokrywają się z deklaracjami współautorów zawartymi w Załączniku 5 do wniosku.
\end{itemize}
\vfill
\newpage

%
% C1
%
\marginnote[1em]{
% MARGIN TAB
\color{red}
\begin{tabular}{rccc}
\toprule
& \textsc{WoS} & \textsc{Sco} & \textsc{GSc}\\
l. cytowań & --- & --- & --- \\\\
\multicolumn{3}{r}{\color{black}\emph{Szacowany udział}} & \color{black}70\%\\
\multicolumn{3}{r}{\color{black}\emph{Impact Factor}} & \color{black}8.139\\
\multicolumn{3}{r}{\color{black}\emph{l. punktów \textsc{me}i\textsc{n}}} & \color{black}200\\\\

%\multicolumn{4}{r}{\includegraphics[width=1.5cm]{WoS.png}}
\end{tabular}

% BASE TAB
}\noindent\begin{tabular}{lllll@{}}
\toprule
\color{red}[C1] & \multicolumn{4}{p{28em}}{\fullcite{C1}} \\
 & \multicolumn{4}{p{30em}}{
 {\footnotesize\textsc{CRediT}}: 
 \mybox{Conceptualization} \mybox{Software} \mybox{Validation} \mybox{Investigation} \mybox{Writing - Original Draft} \mybox{Writing - Review \& Editing} \mybox{Visualization} \mybox{Supervision}}\\

%\\\\\bottomrule
\end{tabular}
\vspace{.5em}

%
% C2
%
\marginnote[1em]{
% MARGIN TAB
\color{red}
\begin{tabular}{rccc}
\toprule
& \textsc{WoS} & \textsc{Sco} & \textsc{GSc}\\
l. cytowań & --- & --- & --- \\\\
\multicolumn{3}{r}{\color{black}\emph{Szacowany udział}} & \color{black}100\%\\
\multicolumn{3}{r}{\color{black}\emph{Impact Factor}} & \color{black}5.086\\
\multicolumn{3}{r}{\color{black}\emph{l. punktów \textsc{me}i\textsc{n}}} & \color{black} 70\\\\

%\multicolumn{4}{r}{\includegraphics[width=1.5cm]{WoS.png}}

\end{tabular}
% BASE TAB
}\noindent\begin{tabular}{lllll@{}}
\toprule
\color{red}[C2] & \multicolumn{4}{p{28em}}{\fullcite{C2}}  \\
 & \multicolumn{4}{p{30em}}{
 {\footnotesize\textsc{CRediT}}: 
 \mybox{Conceptualization} \mybox{Methodology} \mybox{Software} \mybox{Validation} \mybox{Formal Analysis} \mybox{Investigation} \mybox{Resources} \mybox{Data Curation} \mybox{Writing - Original Draft} \mybox{Writing - Review \& Editing} \mybox{Visualization}}\\
%\\\\\bottomrule
\end{tabular}
\vspace{2.5em}

%
% C3
%
\marginnote[1em]{
% MARGIN TAB
\color{red}
\begin{tabular}{rccc}
\toprule
& \textsc{WoS} & \textsc{Sco} & \textsc{GSc}\\
l. cytowań & 2 & 1 & 4 \\\\
\multicolumn{3}{r}{\color{black}\emph{Szacowany udział}} & \color{black}70\%\\
\multicolumn{3}{r}{\color{black}\emph{Core}} & \color{black}B\\
\multicolumn{3}{r}{\color{black}\emph{l. punktów \textsc{me}i\textsc{n}}} & \color{black} 140\\\\

%\multicolumn{4}{r}{\includegraphics[width=1.5cm]{WoS.png}}

\end{tabular}
% BASE TAB
}\noindent\begin{tabular}{lllll@{}}
\toprule
\color{red}[C3] & \multicolumn{4}{p{28em}}{\fullcite{C3}}  \\
 & \multicolumn{4}{p{30em}}{
 {\footnotesize\textsc{CRediT}}: 
 \mybox{Conceptualization} \mybox{Validation} \mybox{Formal Analysis} \mybox{Investigation} \mybox{Resources} \mybox{Writing - Original Draft} \mybox{Writing - Review \& Editing} \mybox{Visualization} \mybox{Supervision}}\\

%\\\\\bottomrule
\end{tabular}
\vspace{.5em}

%
% C4
%
\marginnote[1em]{
% MARGIN TAB
\color{red}
\begin{tabular}{rccc}
\toprule
& \textsc{WoS} & \textsc{Sco} & \textsc{GSc}\\
l. cytowań & 2 & 2 & 4 \\\\
\multicolumn{3}{r}{\color{black}\emph{Szacowany udział}} & \color{black}100\%\\
\multicolumn{3}{r}{\color{black}\emph{Impact Factor}} & \color{black}5.719\\
\multicolumn{3}{r}{\color{black}\emph{l. punktów \textsc{me}i\textsc{n}}} & \color{black} 140\\\\

%\multicolumn{4}{r}{\includegraphics[width=1.5cm]{WoS.png}}

\end{tabular}

% BASE TAB
}\noindent\begin{tabular}{lllll@{}}
\toprule
\color{red}[C4] & \multicolumn{4}{p{28em}}{\fullcite{C4}}  \\
 & \multicolumn{4}{p{30em}}{
 {\footnotesize\textsc{CRediT}}: 
 \mybox{Conceptualization} \mybox{Methodology} \mybox{Software} \mybox{Validation} \mybox{Formal Analysis} \mybox{Investigation} \mybox{Resources} \mybox{Data Curation} \mybox{Writing - Original Draft} \mybox{Writing - Review \& Editing} \mybox{Visualization}}\\

%\\\\\bottomrule
\end{tabular}
\vspace{2.5em}
%\newpage

%
% C5
%
\marginnote[1em]{
% MARGIN TAB
\color{red}
\begin{tabular}{rccc}
\toprule
& \textsc{WoS} & \textsc{Sco} & \textsc{GSc}\\
l. cytowań & 6 & 7 & 13 \\\\
\multicolumn{3}{r}{\color{black}\emph{Szacowany udział}} & \color{black}50\%\\
\multicolumn{3}{r}{\color{black}\emph{Core}} & \color{black}A\\
\multicolumn{3}{r}{\color{black}\emph{l. punktów \textsc{me}i\textsc{n}}} & \color{black} 140\\\\

%\multicolumn{4}{r}{\includegraphics[width=1.5cm]{WoS.png}}
\end{tabular}

% BASE TAB
}\noindent\begin{tabular}{lllll@{}}
\toprule
%\color{red}[C5] & \multicolumn{4}{p{28em}}{\fullcite{C5}}  \\
\color{red}[C5] & \multicolumn{4}{p{28em}}{Paweł Ksieniewicz, Paweł Zyblewski, Michał Choraś, Rafał Kozik, Agata Giełczyk, Michał Woźniak, \emph{"Fake News Detection from Data Streams"}. W: \emph{2020 International Joint Conference on Neural Networks (IJCNN)}. 2020, s. 1-8. \textsc{doi}: \verb|\url{10.1109/IJCNN48605.2020.9207498}}  \\
 & \multicolumn{4}{p{30em}}{
 {\footnotesize\textsc{CRediT}}: 
 \mybox{Conceptualization} \mybox{Methodology} \mybox{Software} \mybox{Validation} \mybox{Investigation} \mybox{Writing - Original Draft} \mybox{Writing - Review \& Editing} \mybox{Visualization}}\\

%\\\\\bottomrule
\end{tabular}
\newpage

%
% C6
%
\marginnote[1em]{
% MARGIN TAB
\color{red}
\begin{tabular}{rccc}
\toprule
& \textsc{WoS} & \textsc{Sco} & \textsc{GSc}\\
l. cytowań & 4 & 5 & 5 \\\\
\multicolumn{3}{r}{\color{black}\emph{Szacowany udział}} & \color{black}100\%\\
\multicolumn{3}{r}{\color{black}\emph{Core}} & \color{black}C\\
\multicolumn{3}{r}{\color{black}\emph{l. punktów \textsc{me}i\textsc{n}}} & \color{black} 20\\\\

%\multicolumn{4}{r}{\includegraphics[width=1.5cm]{WoS.png}}

\end{tabular}

% BASE TAB
}\noindent\begin{tabular}{lllll@{}}
\toprule
\color{red}[C6] & \multicolumn{4}{p{28em}}{\fullcite{C6}}  \\
 & \multicolumn{4}{p{30em}}{
 {\footnotesize\textsc{CRediT}}: 
 \mybox{Conceptualization} \mybox{Methodology} \mybox{Software} \mybox{Validation} \mybox{Formal Analysis} \mybox{Investigation} \mybox{Resources} \mybox{Data Curation} \mybox{Writing - Original Draft} \mybox{Writing - Review \& Editing} \mybox{Visualization}}\\

%\\\\\bottomrule
\end{tabular}
\vspace{2.5em}

%
% C7
%
\marginnote[1em]{
% MARGIN TAB
\color{red}
\begin{tabular}{rccc}
\toprule
& \textsc{WoS} & \textsc{Sco} & \textsc{GSc}\\
l. cytowań & 11 & 18 & 28 \\\\
\multicolumn{3}{r}{\color{black}\emph{Szacowany udział}} & \color{black}50\%\\
\multicolumn{3}{r}{\color{black}\emph{Impact Factor}} & \color{black}4.438\\
\multicolumn{3}{r}{\color{black}\emph{l. punktów \textsc{me}i\textsc{n}}} & \color{black} 140\\\\

%\multicolumn{4}{r}{\includegraphics[width=1.5cm]{WoS.png}}

\end{tabular}
% BASE TAB
}\noindent\begin{tabular}{lllll@{}}
\toprule
%\color{red}[C7] & \multicolumn{4}{p{28em}}{\fullcite{C7}}  \\
\color{red}[C7] & \multicolumn{4}{p{28em}}{Paweł Ksieniewicz, Michał Woźniak, Bogusław Cyganek, Andrzej Kasprzak i Krzysztof Walkowiak. \emph{"Data stream classification using active learned neural networks"}. W: \emph{Neurocomputing 353 (2019)}, s. 74-82. \textsc{doi}: \verb|\url{10.1016/j.neucom.2018.05.130}}  \\
 & \multicolumn{4}{p{30em}}{
 {\footnotesize\textsc{CRediT}}: 
 \mybox{Methodology} \mybox{Software} \mybox{Validation} \mybox{Investigation} \mybox{Writing - Original Draft} \mybox{Writing - Review \& Editing} \mybox{Visualization}}\\

%\\\\\bottomrule
\end{tabular}
\vspace{2.5em}


%
% C8
%
\marginnote[1em]{
% MARGIN TAB
\color{red}
\begin{tabular}{rccc}
\toprule
& \textsc{WoS} & \textsc{Sco} & \textsc{GSc}\\
l. cytowań & --- & --- & 10 \\\\
\multicolumn{3}{r}{\color{black}\emph{Szacowany udział}} & \color{black}100\%\\
\multicolumn{3}{r}{\color{black}\emph{Core}} & \color{black}A\\
\multicolumn{3}{r}{\color{black}\emph{l. punktów \textsc{me}i\textsc{n}}} & \color{black} 140\\\\

%\multicolumn{4}{r}{\includegraphics[width=1.5cm]{WoS.png}}

\end{tabular}
% BASE TAB
}\noindent\begin{tabular}{lllll@{}}
\toprule
\color{red}[C8] & \multicolumn{4}{p{28em}}{\fullcite{C8}}  \\
 & \multicolumn{4}{p{30em}}{
 {\footnotesize\textsc{CRediT}}: 
 \mybox{Conceptualization} \mybox{Methodology} \mybox{Software} \mybox{Validation} \mybox{Formal Analysis} \mybox{Investigation} \mybox{Resources} \mybox{Data Curation} \mybox{Writing - Original Draft} \mybox{Writing - Review \& Editing} \mybox{Visualization}}\\

%\\\\\bottomrule
\end{tabular}
\vspace{2.5em}

%
% C9
%
\marginnote[1em]{
% MARGIN TAB
\color{red}
\begin{tabular}{rccc}
\toprule
& \textsc{WoS} & \textsc{Sco} & \textsc{GSc}\\
l. cytowań & 6 & 11 & 13 \\\\
\multicolumn{3}{r}{\color{black}\emph{Szacowany udział}} & \color{black}80\%\\
\multicolumn{3}{r}{\color{black}\emph{Core}} & \color{black}B\\
\multicolumn{3}{r}{\color{black}\emph{l. punktów \textsc{me}i\textsc{n}}} & \color{black} 15\\\\

%\multicolumn{4}{r}{\includegraphics[width=1.5cm]{WoS.png}}

\end{tabular}
% BASE TAB
}\noindent\begin{tabular}{lllll@{}}
\toprule
\color{red}[C9] & \multicolumn{4}{p{28em}}{\fullcite{C9}}  \\
 & \multicolumn{4}{p{30em}}{
 {\footnotesize\textsc{CRediT}}: 
 \mybox{Methodology} \mybox{Software} \mybox{Validation} \mybox{Investigation} \mybox{Writing - Original Draft} \mybox{Writing - Review \& Editing} \mybox{Visualization}}\\

%\\\\\bottomrule
\end{tabular}
\newpage

%
% C10
%
\marginnote[1em]{
% MARGIN TAB
\color{red}
\begin{tabular}{rccc}
\toprule
& \textsc{WoS} & \textsc{Sco} & \textsc{GSc}\\
l. cytowań & 15 & 15 & 18 \\\\
\multicolumn{3}{r}{\color{black}\emph{Szacowany udział}} & \color{black}60\%\\
\multicolumn{3}{r}{\color{black}\emph{Impact Factor}} & \color{black}4.072\\
\multicolumn{3}{r}{\color{black}\emph{l. punktów \textsc{me}i\textsc{n}}} & \color{black} 30\\\\

%\multicolumn{4}{r}{\includegraphics[width=1.5cm]{WoS.png}}

\end{tabular}
% BASE TAB
}\noindent\begin{tabular}{lllll@{}}
\toprule
\color{red}[C10] & \multicolumn{4}{p{28em}}{\fullcite{C10}}  \\
 & \multicolumn{4}{p{30em}}{
 {\footnotesize\textsc{CRediT}}: 
 \mybox{Conceptualization} \mybox{Methodology} \mybox{Software} \mybox{Validation} \mybox{Writing - Original Draft} \mybox{Writing - Review \& Editing} \mybox{Visualization}}\\

%\\\\\bottomrule
\end{tabular}
\vspace{.5em}

%
% C11
%
\marginnote[1em]{
% MARGIN TAB
\color{red}
\begin{tabular}{rccc}
\toprule
& \textsc{WoS} & \textsc{Sco} & \textsc{GSc}\\
l. cytowań & --- & --- & 9 \\\\
\multicolumn{3}{r}{\color{black}\emph{Szacowany udział}} & \color{black}80\%\\
\multicolumn{3}{r}{\color{black}\emph{Core}} & \color{black}A\\
\multicolumn{3}{r}{\color{black}\emph{l. punktów \textsc{me}i\textsc{n}}} & \color{black} 140\\\\

%\multicolumn{4}{r}{\includegraphics[width=1.5cm]{WoS.png}}

\end{tabular}
% BASE TAB
}\noindent\begin{tabular}{lllll@{}}
\toprule
\color{red}[C11] & \multicolumn{4}{p{28em}}{\fullcite{C11}}  \\
 & \multicolumn{4}{p{30em}}{
 {\footnotesize\textsc{CRediT}}: 
 \mybox{Conceptualization} \mybox{Methodology} \mybox{Software} \mybox{Validation} \mybox{Formal Analysis} \mybox{Investigation} \mybox{Resources} \mybox{Data Curation} \mybox{Writing - Original Draft} \mybox{Writing - Review \& Editing} \mybox{Visualization}}\\

\end{tabular}


\vfill

\subsection{Informacje naukometryczne}
Podane poniżej wartości oddają \textbf{stan na dzień 27. sierpnia 2022 r.} zgodnie z bazami publikacji naukowych \emph{Web of Science} (\textsc{WoS}), \emph{Scopus} (\textsc{Sco}) oraz \emph{Google Scholar} (\textsc{GSc}).

\vspace{4em}
\renewcommand{\arraystretch}{1.1}
\noindent\begin{tabular}{p{.9em}llrrr}
%\toprule
$\circ$ & \multicolumn{4}{l}{Sumaryczny IF dla osiągnięcia} & \multicolumn{1}{r}{27,454}\\

$\circ$ & \multicolumn{2}{l}{Sumaryczne MEiN} & & & \multicolumn{1}{r}{1~175}\\\\

& & & \textsc{\emph{WoS}} & \textsc{Sco} & \textsc{GSc}\\

$\circ$ & \multicolumn{2}{l}{Sumaryczna liczba cytowań dla osiągnięcia} & 46 & 59 & 104\\
$\circ$ & \multicolumn{2}{l}{Indeks Hirscha autora} & 8 & 10 & 13\\
 & & \multicolumn{1}{r}{\emph{l. cytowań.}} & 157 & 276 & 377\\
 & & \multicolumn{1}{r}{\emph{bez autocytowań}} & 134 & 240 & ---\\
 & & \multicolumn{1}{r}{\emph{l. dokumentów}} & 36 & 42 & 51\\
\end{tabular}


\vfill
