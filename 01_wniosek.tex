\newpage
\thispagestyle{empty}
\begin{fullwidth}
	
\hspace{.75\textwidth}
\begin{minipage}{.65\textwidth}
\textbf{Politechnika Wrocławska}\\
Wybrzeże Wyspiańskiego 27\\
50-370 Wrocław\\
Dyscyplina naukowa: \textbf{Informatyka Techniczna i Telekomunikacja}\\
\underline{za pośrednictwem:}\\
\textbf{Rady Doskonałości Naukowej}\\
pl. Defilad 1\\
00-901 Warszawa\\
(Pałac Kultury i Nauki, p. XXIV, pok. 2401)
\end{minipage}

\noindent\begin{minipage}{.75\textwidth}
\textbf{Paweł Ksieniewicz}\\
\textbf{Politechnika Wrocławska}\\
\textbf{Wydział Informatyki i Telekomunikacji}\\
\textbf{Katedra Systemów i Sieci Komputerowych}\\
Wybrzeże Wyspiańskiego 27\\
50-370 Wrocław
\end{minipage}
\vspace{1em}

\begin{center}
	{\textbf{\large Wniosek}}
	
	z dnia 27.08.2022	
\end{center}

o przeprowadzenie postępowania w sprawie nadania stopnia doktora habilitowanego w dziedzinie \textbf{Nauk inżynieryjno-technicznych} w dyscyplinie$^1$ \textbf{Informatyka techniczna i telekomunikacja}.\vspace{1em}

\hspace{3em}\begin{minipage}{45em}
Określenie osiągnięcia naukowego będącego podstawą ubiegania się o nadanie stopnia doktora habilitowanego: cykl publikacji naukowych zatytułowany „\textbf{Projektowanie algorytmów rozpoznawania wzorców dla zadania klasyfikacji trudnych danych}”.
\end{minipage}\vspace{1em}

\noindent Wnioskuję -- na podstawie art. 221 ust. 10 ustawy z dnia 20 lipca 2018 r. Prawo o szkolnictwie wyższym i nauce (Dz. U. z 2021 r. poz. 478 zm.) – aby komisja habilitacyjna podejmowała uchwałę w~sprawie nadania stopnia doktora habilitowanego w głosowaniu \textbf{jawnym}$^{*2}$.\vspace{1em}

\begin{minipage}{45em}
\it
Zostałem poinformowany, że:

Administratorem w odniesieniu do danych osobowych pozyskanych w ramach postępowania w sprawie nadania stopnia doktora habilitowanego jest Przewodniczący Rady Doskonałości Naukowej z siedzibą w Warszawie (pl. Defilad 1, XXIV piętro, 00-901 Warszawa).

Kontakt za pośrednictwem e-mail: \url{kancelaria@rdn.gov.pl}, tel. 22 656 60 98 lub w siedzibie organu. Dane osobowe będą przetwarzane w oparciu o przesłankę wskazaną w art. 6 ust. 1 lit. c) Rozporządzenia UE 2016/679 z dnia z dnia 27 kwietnia 2016 r. w związku z art. 220 - 221 orazart. 232 – 240 ustawy z dnia 20 lipca 2018 roku - Prawo o szkolnictwie wyższym i nauce, w celu przeprowadzenie postępowania o nadanie stopnia doktora habilitowanego oraz realizacji praw i obowiązków oraz środków odwoławczych przewidzianych w tym postępowaniu.

Szczegółowa informacja na temat przetwarzania danych osobowych w postępowaniu dostępna jest na stronie \url{www.rdn.gov.pl/klauzula-informacyjna-rodo.html}
\end{minipage}

\vspace{5em}
\vfill\noindent\hspace{.75\textwidth}\begin{minipage}{15em}
\begin{center}
	\hbox to 5cm{\leaders\hbox to 3pt{\hss . \hss}\hfil}

	(podpis wnioskodawcy)
\end{center}
\end{minipage}

\vfill\noindent\begin{minipage}{50em}
\footnotesize\rule{15em}{.5pt}

$^{1}$ Klasyfikacja dziedzin i dyscyplin wg. rozporządzenia Ministra Nauki i Szkolnictwa Wyższego z dnia 20 września 2018 r. w sprawie dziedzin nauki i dyscyplin naukowych oraz dyscyplin w zakresie sztuki (Dz. U. z 2018 r. poz. 1818).

$^{2*}$ Niepotrzebne skreślić.
	
\end{minipage}

\newpage
\thispagestyle{empty}

\underline{Załączniki}:
\begin{enumerate}
	\item Dane wnioskodawcy
	\item Kopia dokumentu potwierdzającego posiadanie stopnia doktora
	\item Autoreferat wnioskodawcy
	\item Wykaz osiągnięć naukowych
	\item Deklaracje współautorów dotyczące wkładu pracy
	\item Publikacje wchodzące w skład osiągnięcia naukowego „Projektowanie algorytmów rozpoznawania wzorców dla zadania klasyfikacji danych trudnych”
	%\item Inne wybrane publikacje naukowe
\end{enumerate}

\end{fullwidth}
