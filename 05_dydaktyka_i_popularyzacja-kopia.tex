
\section{Informacja o osiągnięciach dydaktycznych, organizacyjnych oraz popularyzujących naukę lub sztukę}

Podczas dziewięciu lat pracy naukowo-dydaktycznej prowadziłem dziewiętnaście kursów dla studentów kierunków Informatyka, Informatyka Techniczna i Teleinformatyka, w siedmiu z nich będąc głównym wykładowcą: 

\begin{enumerate}
	\item Metody sztucznej inteligencji (laboratorium, \textbf{wykład}, projekt),
	\item Methods of Computational Intelligence and Decision Making (laboratorium, projekt, \textbf{wykład}),
	\item Obrazowanie biomedyczne (laboratorium, seminarium, \textbf{wykład}),
	\item Przetwarzanie sygnałów wielowymiarowych (\textbf{wykład}, laboratorium),
	%\item Współczesne trendy w informatyce (seminarium, \textbf{wykład}),
	\item Projektowanie systemów internetowych i mobilnych (projekt, seminarium, \textbf{wykład}),
	\item Projektowanie telemedycznych systemów internetowych i mobilnych (laboratorium, \textbf{wykład}),
	%\item Aplikacje mobilne (laboratorium, \textbf{wykład}),\vspace{1em}
	%\item Research Skills and Methodologies (laboratorium, projekt)
	%\item Zastosowanie informatyki w medycynie (projekt, seminarium),
	%\item Projektowanie efektywnych algorytmów (projekt),
	%\item Seminarium dyplomowe (seminarium),\vspace{1em}
	%\item Programowanie Obiektowe (laboratorium),
	%\item Bazy danych (laboratorium),
	%\item Systemy operacyjne (laboratorium),
	%\item Computer Project Management (laboratorium),
	%\item Cyfrowe przetwarzanie sygnałów i obrazów (laboratorium),
	%\item Inżynierskie zastosowanie statystyki (ćwiczenia),
	%\item Projekt zespołowy (projekt),
\end{enumerate}

Angażuję się również w opracowywanie materiałów dydaktycznych dla studentów, uczniów oraz nauczycieli. Przygotowałem materiały multimedialne do nauki sztucznej inteligencji oraz serię materiałów wideo dla projektu \emph{Centrum Mistrzostwa Informatycznego}. Współorganizowałem również kilka edycji studenckich warsztatów naukowych International Students Workshop (2015-2019), Hackathon JellyPizzaHack organizowany we współpracy z Credit Suisse (16.12.2016) oraz opiekuję się pracami badawczymi studentów w ramach działającego przy Katedrze Systemów i Sieci Komputerowych, Koła Naukowego Systemów i Sieci Komputerowych oraz Koła Uczenia Maszyn. Aktywnie uczestniczę również w opracowywaniu kart przedmiotów, będąc opiekunem  czterech kursów:

\begin{itemize}
	\item Projektowanie systemów informatyki medycznej (INES00123),
	\item Przetwarzanie sygnałów wielowymiarowych (INEU00127),
	\item Metody przetwarzania języka naturalnego oraz wyszukiwanie (INEU00129),
	\item Uczenie Maszyn (INEA00236).
\end{itemize}

Od pięciu lat pełnię także rolę sekretarza komisji dyplomowej specjalności \emph{Advanced Informatics and Control}, prowadzonej w języku angielskim. W tym czasie byłem także promotorem 43 prac magisterskich i 37 prac inżynierskich. Sześcioro z moich dyplomantów aktualnie realizuje swoje prace doktorskie na Politechnice Wrocławskiej, a z pięciorgiem z nich miałem przyjemność pracować przy ich pierwszych publikacjach konferencyjnych lub publikacjach w czasopismach. Poniżej prezentuję wybraną tematykę prowadzonych prac magisterskich z zaznaczeniem obecnych doktorantów:

\begin{itemize}
	\item Prior Probability Estimation in Dynamically Imbalanced Data Streams, 2022,\\\emph{mgr inż. Joanna Komorniczak},
	%\item Analiza wpływu wykorzystania metod uczenia nadzorowanego na procedurę optymalizacyjną problemu komiwojażera, 2022,
	%\item Detekcja manipulacji w materiale wideo z wykorzystaniem głębokich sieci neuronowych, 2021
	%\item Fragmentaryczna detekcja i klasyfikacja twarzy z wykorzystaniem metod uczenia głębokiego, 2021
%	\item Wykorzystanie głębokich sieci konwolucyjnych w problemie klasyfikacji binarnej chorób płuc, 2021,
%	\item Wykorzystanie klasycznych metod rozpoznawania wzorców w klasyfikacji danych medycznych, 2021,
	\item Feature extraction using n-gram methods for the purpose of ensemble classification of disinformation sources, 2021,\\\emph{mgr inż. Weronika Borek-Marciniec}
%	\item Detection and multi-label classification of objects in hand drawings, 2021,
	%\item Detekcja manipulacji zawartości zdjęć przy pomocy metod uczenia głębokiego, 2020,
%	\item Analiza porównawcza najpopularniejszych architektur konwolucyjnych sieci neuronowych w problemie klasyfikacji obrazów, 2020,
%	\item Analiza porównawcza architektonicznych wzorców projektowych aplikacji internetowych na przykładzie REST i GraphQL, 2020,
	%\item Wykorzystanie algorytmów probabilistycznego rozpoznawania wzorców w problemie klasyfikacji binarnej tzw. fake news, 2020,
%	\item Porównanie algorytmów wyszukiwania ścieżek w przestrzeni dwuwymiarowej, 2020,
	%\item Analiza wpływu trendów wyszukiwania w predykcji wartości wybranej kryptowaluty z wykorzystaniem algorytmów regresji, 2020,
%	\item System szybkiej pomocy w rozwiązywaniu zadań dla szkół licealnych, 2020
%	\item Analiza porównawcza wybranych serwerowych platform programistycznych aplikacji internetowych, 2020,
	\item Wykorzystanie uczenia zespołowego w klasyfikacji binarnej niezbalansowanych strumieni danych, 2020\\\emph{mgr inż. Weronika Węgier}
%	\item Metody detekcji i rozpoznawania znaków drogowych, 2019,
    %\item Application of pattern recognition techniques for customer revenue prediction, 2019,
%	\item Analysis of synthetic oversampling methods for the problem of imbalanced data streams, 2019,
%	\item Audio-visual system of a surrounding analysis assisting people with visual disturbance in sense of direction, 2019,
	\item Analiza efektywności odmian algorytmu SMOTE w balansowaniu strumieni danych, 2019,\\\emph{mgr inż. Bogdan Gulowaty},
%	\item Oversampling niezbalansowanych strumieni danych na potrzeby zadania klasyfikacji, 2019,
	\item Analiza efektywności zastosowania sieci rekurencyjnych w zadaniu klasyfikacji, 2019,\\\emph{mgr inż. Jędrzej Kozal},
	%\item Projekt i implementacja systemu wieloetykietowej klasyfikacji instrumentów muzycznych w nagraniach dźwiękowych, 2019,
%	\item Analiza porównawcza wydajności silników baz danych w zadaniu przechowywania danych medycznych., 2018,
	%\item Wykorzystanie detektorów anomalii rozwijających algorytm RX dla obrazów nadwidmowych, 2018,
	%\item Efficiency analysis of data stream classifier ensemble pruning methods, 2018,
%	\item Effectivity analysis of distance measurement algorithms based on target detection and tracking, 2018,
	%\item Optimizing the process of generating concept drift data streams, 2018,
%	\item Mobile network standards coverage analysis for selected residentials in Wroclaw, 2018,
	%\item The binary classification of imbalanced data streams, 2018,
%	\item Efficiency of the microservices implementation on the example of the cloud applications, 2018,
%	\item Comparative analysis of the efficiency of databases types in the aspect of reactive programming, 2018,
%	\item Analiza efektywności wykorzystania zespołu detektorów w problemie detekcji dryfu danych strumieniowych, 2018,
%	\item Analiza porównawcza efektywności interfejsów programistycznych zrealizowanych w architekturze REST i GrapQL, 2018,
%	\item Wizyjno-sensoryczny system wspomagania parkowania na urządzenie mobilne, 2018,
	\item Zadanie uczenia nienadzorowanego w kontekście obrazowania nadwidmowego, 2018,\\\emph{mgr inż. Dominika Sułot}
\end{itemize}

Aktywnie udzielam się również w organizacji konferencji i warsztatów naukowych poświęconych zagadnieniom klasyfikacj danych trudnych, wśród których chciałbym wymienić:

\begin{itemize}
    \item \ Organizację sesji specjalnej “Machine Learning Algorithms for Hard Problems“ na konferencji 20th International Conference on Intelligent Data Engineering and Automated Learning (IDEAL), 14–16 listopada 2019, Manchester, Anglia. Zasięg międzynarodowy.
    \item Organizację sesji specjalnej “Classifier Learning from Difficult Data” na konferencji INTERNATIONAL CONFERENCE ON COMPUTATIONAL SCIENCE, 3–5 czerwca 2020, Amsterdam, Holandia. Zasięg międzynarodowy.
    \item Organizację sesji specjalnej “Classifier Learning from Difficult Data” na konferencji INTERNATIONAL CONFERENCE ON COMPUTATIONAL SCIENCE, 12–14 czerwca 2019, Faro, Portugalia. Zasięg międzynarodowy.
    \item Organizację sesji specjalnej “Classifier Learning from Difficult Data” na konferencji INTERNATIONAL CONFERENCE ON COMPUTATIONAL SCIENCE, 12–14 czerwca 2019, Faro, Portugalia. Zasięg międzynarodowy.
    \item Organizację konferencji Polskie Porozumienie na rzecz Rozwoju Sztucznej Inteligencji, 16– 18 października 2019, Wrocław, Polska. Zasięg krajowy.
    \item Organizację konferencji The 9 International Conference on Computer Recognition Systems, CORES 2015, Wrocław, Polska. Zasięg międzynarodowy
\end{itemize}

Wielokrotnie udzielałem się także w działaniach mających na celu zwiększenie świadomości społecznej z zakresu ryzyka związanego z dezinformacją i potencjału metod sztucznej inteligencji w walce z nią. Należy wymienić tu:

\begin{itemize}
    \item uczestnictwo w roli eksperta w debacie “Szczepionka na kłamstwo.” organizowanej przez Fundację na rzecz Nauki Polskiej w ramach cyklu „Ufajmy nauce” (21.04.2022r.),
    \item prelekcję “Wykorzystanie sztucznej inteligencji w walce z dezinformacją” w ramach seminarium "Sztuczna inteligencja w rozwoju miast i obszarów metropolitarnych" organizowanego przez Wrocławskie Centrum Akademickie pod patronatem World Urban Forum (9.03.2022r.),
    \item wykład w ramach seminarium “Machine Learning to Combat Fake News and Media Manipulation” organizowanego przez Elsevier w ramach cyklu webinarów (20.04.2022r.).
    \item wywiady radiowe dotyczące detekcji źródeł dezinformacji.
\end{itemize}

W ostatnich latach przeprowadziłem także trzy wykłady na zaproszenie, dotyczące tematyki klasyfikacji danych trudnych oraz detekcji źródeł dezinformacji:

\begin{itemize}
    \item Wykład na zaproszenie:\\
    Research practices in data stream analysis and imbalanced data classification.\\
    22 czerwca 2020, Amity School of Engineering and Technology, Noida, Indie,
    \item Wykład w ramach Elsevier Webinar Machine Learning to combat Fake News and Media Manipulation.\\
    Using machine learning as the weapon against the disinformation\\
    20 kwietnia 2021\\
    \url{https://www.workcast.com/register?cpak=7948916184707381}
    \item Keynote podczas sesji specjalnej CLDD w ramach konferencji International Conference on Computational Science.\\
    Chosen Challenges of Imbalanced Data Stream Classification\\
    16 czerwca 2021\\
    \url{https://www.iccs-meeting.org/iccs2021/}
\end{itemize}
