\section{Imię i nazwisko}

\vspace{-.5em}Paweł Ksieniewicz\vspace{-1em}

\section{Posiadane dyplomy, stopnie naukowe lub artystyczne -- z podaniem podmiotu nadającego stopień, roku ich uzyskania oraz tytułu rozprawy doktorskiej}

\vspace{-.5em}\begin{tabular}{lrp{25em}}
\textbf{\oldstylenums{2017}} & 	\multicolumn{2}{p{30em}}{\textbf{Stopień doktora nauk technicznych w dyscyplinie Informatyka.}}\\
&\multicolumn{2}{l}{\emph{Nadany uchwałą Rady Wydziału Elektroniki Politechniki Wrocławskiej.}}\\\\
&Tytuł rozprawy: & \emph{Multidimensional data representation and analysis}\\
&Tytuł w języku polskim:&\emph{Reprezentacja i analiza danych wielowymiarowych}\\
&Promotor: &\emph{prof. dr hab. inż. Michał Woźniak}\\\\

\textbf{\oldstylenums{2017}} & 	\multicolumn{2}{p{30em}}{\textbf{Tytuł zawodowy magistra inżyniera w dyscyplinie Informatyka.}}\\
&\multicolumn{2}{l}{\emph{Wydział Elektroniki Politechniki Wrocławskiej.}}\\\\
&Tytuł pracy magisterskiej: & \emph{System wykrywania sztormów na Morzu Bałtyckim z wykorzystaniem metorogramów}\\
&Promotorka: &\emph{dr inż. Iwona Poźniak-Koszałka}\\\\
\end{tabular}\vspace{-1em}

\section{Informacje o dotychczasowym zatrudnieniu w jednostkach naukowych lub artystycznych}

\vspace{-.5em}\begin{tabular}{r@{}c@{}lp{40em}}
\bfseries 10.2017 & \bfseries~--~ & \bfseries obecnie & \textbf{Adiunkt}\\

&&& Katedra Systemów i Sieci Komputerowych\\
&&& Wydział Informatyki i Telekomunikacji\\
&&& Politechnika Wrocławska\\
&&& \emph{(do roku 2021 w Katedrze Systemów i Sieci Komputerowych Wydziału Elektroniki PWr)}\\\\

\bfseries 5.2019 & \bfseries~--~ & \bfseries 11.2021 & \textbf{Specjalista ds. sztucznej inteligencji}}\\

&&& Uniwersytet Technologiczno-Przyrodniczy im. Jana i Jędrzeja Śniadeckich w Bydgoszczy\\\\
%&&& Wydział Elektroniki\\
%&&& Politechnika Wrocławska\\

\bfseries 2.2015 & \bfseries~--~ & \bfseries 10.2017 & \textbf{Asystent}\\

&&& Katedra Systemów i Sieci Komputerowych\\
&&& Wydział Elektroniki\\
&&& Politechnika Wrocławska\\

\end{tabular}\vspace{-1em}


%\newpage

% 3.4
\section{Omówienie osiągnięć, o których mowa w art. 219 ust. 1 pkt. 2 ustawy z dnia 20 lipca 2018 r. Prawo o szkolnictwie wyższym i nauce (Dz. U. z 2021 r. poz. 478 z późn. zm.)}

